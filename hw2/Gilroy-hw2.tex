\documentclass[12pt,letterpaper,boxed]{hmcpset}
\usepackage[margin=1in,headheight=14pt]{geometry}
\usepackage{amsfonts, amsmath, amssymb, enumerate, fancyhdr, gensymb, lastpage, mathtools, parskip, graphicx}
\usepackage{xcolor, tikz-cd}
\newcommand{\wg}[1]{\textcolor{violet}{#1}}
\newcommand{\OO}{\mathcal O}
\newcommand{\Q}{\mathbb Q}
\newcommand{\R}{\mathbb R}
\newcommand{\C}{\mathcal C}
\newcommand{\Z}{\mathbb Z}
\newcommand{\abs}[1]{\left|#1\right|}
\newcommand{\im}{\text{im }}
\newcommand{\tih}{\tilde h}
\newcommand{\inv}{^{-1}}
\newcommand{\normal}{\unlhd} %% one can also use \trianglelelefteq
\usepackage[shortlabels]{enumitem}

% Numbering macros
\pagestyle{fancy}
\lhead{Will Gilroy}
\chead{Algs Homework \#}
\rhead{03 November 2021}
\lfoot{}
\cfoot{}
\rfoot{Page\ \thepage\ of\ \pageref{LastPage}}

\linespread{1.5}

\newcommand\blankpage{
    \thispagestyle{empty}
    \addtocounter{page}{-1}
    \newpage}
\renewcommand\footrulewidth{0.4pt}

\begin{document}

\problemlist{Algorithms HW } 

%------------------------- Problem 1 -----------------------

\begin{problem}[1]
	\hfill
\end{problem}

\begin{solution}
\end{solution}

\newpage

%------------------------- Problem 2 -----------------------

\begin{problem}[2]
	\hfill
\end{problem}

\begin{solution}
\end{solution}

\newpage

%------------------------- Problem 3 -----------------------

\begin{problem}[3]
	\hfill
\end{problem}
\begin{solution}
\end{solution}

\newpage

%------------------------- Problem 4 -----------------------

\begin{problem}
	\includegraphics[scale=0.8]{4.png}
	\hfill
\end{problem}

\begin{solution}
Firstly, we will show that $\rho$ is a well defined map $H \to
Out(N)$. Let $h \in H$ and $\tih_1, \tih_2 \in G$ such that $\pi(\tih_1)
= \pi(\tih_2) = h$. We have 
$\rho(\tih_1) = f := (n \mapsto \tih_1 n \tih_1\inv)$ and
$\rho(\tih_2) = g := (n \mapsto \tih_2 n \tih_2\inv)$. 
Note that these are indeed automorphisms of $N$, as in the previous
homework we showed that conjugation by a fixed element is an
automorphism.
If we show that $\rho(\tih_1)$ and $\rho(\tih_2)$ lie in the same
coset of $Inn(N)$ then $\rho$ is well-defined. (Note: I believe this
map is not well defined as a map $H \to Aut(N)$). 

Recall that two elements $g,h$ of a group lie in the same coset of a normal
subgroup $N$ if $g\inv h \in N$. For our automorphisms $f,g$ we have
$g\inv = (n \mapsto \tih_2\inv n \tih_2)$. And so we have 
$(g\inv \circ f)(n) = \tih_2\inv \tih_1 n \tih_1\inv \tih_2$. Recall
that $N \normal G$ and so is closed under conjugation by definition.
In particular then $\tih_1 n \tih_1\inv \in N$ and $\tih_2\inv(\tih_1
n \tih_1 \inv) \tih_2 \in N$ since $\tih_1, \tih_2 \in G$. 
Thus $f,g$ have the same image in $Out(N)$ and so $\rho$ is well
defined with respect to the choice of $\tih$. 

Next we show that $\rho$ is a group homomorphism. Let $h_1, h_2 \in H$
and $\tih_1, \tih_2 \in G$ such that $\pi(\tih_1) = h_1$ and
$\pi(\tih_2) = h_2$. Moreover, since $\pi$ is a group homomorphism we
have $\pi(\tih_1\tih_2) = \tih_1 \tih_2$. Following a similar,
calculation to last week's homework, consider the following

\begin{align*}
	\rho(h_1h_2) &= \gamma_{\tih_1\tih_2} \\
		&= (n \mapsto \tih_1\tih_2 n (\tih_1\tih_2)\inv) \\
		&= (n \mapsto \tih_1 \tih_2 n \tih_2\inv \tih_1\inv) \\
		&= \gamma_{\tih_1} \circ \gamma_{\tih_2} \\
		&= \rho(h_1)\rho(h_2).
\end{align*}
Thus, the given $\rho$ is indeed a group homomorphism.

Now suppose $G = N \rtimes_\theta H$. We can state more precisely the
outer automorphism given by $\rho$. Let $h \in H$ and then all lifts
are of the form $\tih = (m, h)$ for some $m \in N$. Then, being
explicit about the details of the semidirect product, our map
$\rho(h) : \iota(N) \to \iota(N)$ acts as follows 
\begin{align*}
	\rho_h(n) &= (m, h) \cdot_\theta (n, e_H) \cdot_\theta (m, h)\inv \\
		&= (m,h) (n, e_H) (\theta_{h\inv}(m\inv), h\inv) \\
		&= (m \theta_h(n), h) (\theta_{h\inv}(m\inv), h\inv) \\
		&= (m \theta_h(n) (\theta_h \circ \theta_{h\inv}(m\inv), h h\inv) \\
		&= (m \theta_h(n) m\inv, e_H).
\end{align*}
Which induces the automorphism $f = (n \mapsto m\theta_h(n)m\inv) : N \to
N$. Note that $(\theta_{h} \theta_{h\inv}) = id_H$ since $\theta$ is a
group homomorphism $H \to Aut(N)$. 

We show that this is the same as the composition $H \to Aut(N) \to
Out(N)$. We have $h \mapsto \theta_h \mapsto \overline{\theta_h}$. 
Notice now that $\theta_h$ and $f$ are lie in the same coset of
$Inn(N)$. In particular \[
	\overline{\theta_h} = \overline{\gamma_m \theta_h} = \overline{f}
\]
since $\gamma_m = (n \mapsto m n m\inv)$ is one of the inner
automorphisms of $N$.
Hence, in the case where $G = N \rtimes_\theta H$ we have $\rho$ and
$H \to Aut(N) \to Out(N)$ give the same map.

One interpretation of this is that, whilst $\rho$ is a well defined
map $H \to Out(N)$, it is not a well defined map $H \to Aut(N)$.
However, in the case where $G$ is a semidirect product of $N$ and $H$
via $\theta$,
we have a preferred lift $h \mapsto (e_N, h) \in G$, 
and in fact there is a well defined map $H \to Aut(N)$, namely
$\theta$, whose projection gives the same map as $\rho$. 


\end{solution}

\newpage
%------------------------- Problem 5 -----------------------

\begin{problem}
    \includegraphics[scale=0.8]{5-1.png}
	\includegraphics[scale=0.8]{5-2.png}
	\hfill
\end{problem}

\begin{solution}
\begin{itemize}
\item Sylow's theorem I gives us that there exists a subgroup of order
$7$ in $G$, since $\abs{H} = 7^1 \cdot 4$ and $7 \not \vert 4$. 
Alternatively, Cauchy's theorem gives us that there exists an element
$g \in G$ with $\abs{g} = 7$, hence we have $\abs{\langle g \rangle}
\leq G$. Moreover, Sylow III gives us that there's only a single Sylow
$7$ group. Consider, if $n_7$ is the number of Sylow $7$ groups in
$G$ then Sylow III gives us that $n_7 \equiv 1 \mod 7$ and $n_7 \vert
4$. The only integer solving both these conditions is $n_p = 1$. 
Lastly, if we let $N$ denote the Sylow $7$ group then
\wg{Normality is somehow acquired from Sylow II and the fact that $N$
is the only subgroup of order $7$, but I don't quite understand how
this works.}
Sylow II gives us that that $N$ is acquired from $N$ by conjugation 

\wg{I'm also not super sure why there's necessarily a subgroup of order $4$}

\end{itemize}
\end{solution}

\newpage
%------------------------- Problem 4 -----------------------

\begin{problem}[4]
	\hfill
\end{problem}

\begin{solution}
\end{solution}

\newpage

%------------------------- Problem 4 -----------------------

\begin{problem}[4]
	\hfill
\end{problem}

\begin{solution}
\end{solution}

\newpage

%------------------------- Problem 4 -----------------------

\begin{problem}[4]
	\hfill
\end{problem}

\begin{solution}
\end{solution}

\newpage



\end{document}
