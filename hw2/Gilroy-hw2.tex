\documentclass[12pt,letterpaper,boxed]{hmcpset}
\usepackage[margin=1in,headheight=14pt]{geometry}
\usepackage{amsfonts, amsmath, amssymb, enumerate, fancyhdr, gensymb, lastpage, mathtools, parskip, graphicx}
\usepackage{xcolor, tikz-cd}
\newcommand{\wg}[1]{\textcolor{violet}{#1}}
\newcommand{\OO}{\mathcal O}
\newcommand{\Q}{\mathbb Q}
\newcommand{\R}{\mathbb R}
\newcommand{\C}{\mathcal C}
\newcommand{\Z}{\mathbb Z}
\newcommand{\abs}[1]{\left|#1\right|}
\newcommand{\im}{\text{im }}
\newcommand{\tih}{\tilde h}
\newcommand{\inv}{^{-1}}
\newcommand{\normal}{\unlhd} %% one can also use \trianglelelefteq
\newcommand{\anglee}[1]{\langle #1 \rangle}
\usepackage[shortlabels]{enumitem}

% Numbering macros
\pagestyle{fancy}
\lhead{Will Gilroy}
\chead{Algebra Homework \#2}
\rhead{17 September 2025}
\lfoot{}
\cfoot{}
\rfoot{Page\ \thepage\ of\ \pageref{LastPage}}

\linespread{1.5}

\newcommand\blankpage{
    \thispagestyle{empty}
    \addtocounter{page}{-1}
    \newpage}
\renewcommand\footrulewidth{0.4pt}

\begin{document}

\problemlist{Math6310: Algebra Homework \#2} 

%------------------------- Problem 1 -----------------------

\begin{problem}
	\includegraphics[scale=0.8]{1.png}
	\hfill
\end{problem}

\begin{solution}
\begin{itemize}
\item 
We show that each element of the descending central series is normal
in $G$  by induction on $i$. Since
$\Gamma^1 G = G$ our base case is $i=2$. Recall that $\Gamma^2 G$ is
generated by commutators $[g,h]$ where $g \in G$, $h \in \Gamma^1G =
G$.
Now let $k \in G$
and the following shows that $\Gamma^2 G$ is closed under conjugation
by elements in $G$ 
\begin{align*}
	k [g,h] k\inv = k g h g\inv h\inv k\inv
	= (kgk\inv)(k h k\inv) (k g\inv k\inv) (k h\inv k\inv)
	= [k g k\inv, k h k\inv] \in \Gamma^2 G.
\end{align*}
And so, $\Gamma^2 G \normal G$ by definition.

Now suppose $\Gamma^i G \normal G$ and consider $\Gamma^{i+1} G$,
which is generated by elements $[g,h]$ where now $g \in G$ and $h \in
\Gamma^i G$. Let $k \in G$ and notice that, since $\Gamma^i G$ is
normal by the induction hypothesis, 
we have $k h k\inv \in \Gamma^iG$. Thus the above calculation
gives that $k[g,h]k \inv = [k g k\inv, k h k\inv] \in \Gamma^{i+1}G$.
Thus, each $\Gamma^i G \normal G$ by induction.

\item 
We show that $\Gamma^{i+1} G \subseteq \Gamma^i G$. Let $[g,h]$ be a
generator of $\Gamma^{i+1}G$, so that $g \in G$ and $h \in
\Gamma^{i}G$. 
Above we showed that $\Gamma^i G \normal G$ and so $ghg\inv \in
\Gamma^i G$. Moreover, $h \in \Gamma^i G$ implies $h\inv \in \Gamma^i
G$, since $\Gamma^i G$ is a group. Thus \[
	[g,h] = ghg\inv h\inv = (g h g\inv) h\inv \in \Gamma^i G. 
\]
Then, since each generator of $\Gamma^{i+1} G$ is contained in
$\Gamma^i G$, we have $\Gamma^{i+1} G \subseteq \Gamma^{i}G$. 

\item \wg{todo}
\end{itemize}

\end{solution}

\newpage

%------------------------- Problem 2 -----------------------

\begin{problem}
	\includegraphics[scale=0.8]{2.png}
	\hfill
\end{problem}

\begin{solution}
Big sorry, I'm going to use $D_n$ to denote the dihedral group which
has $2n$ elements. Recall the presentation \[
	D_n = \langle 
		r,s : r^n = s^2 = e \quad rs = sr\inv 
	\rangle
\]

First note that $D_1$ and $D_2$ are abelian. 
We have $D_1 = \{ e, s \}$ a single non-trivial element, and so is
trivially abelian. Meanwhile $D_2 = \{ e, s, r, rs \}$, the relation 
$rs = sr\inv$ becomes $rs = sr$, i.e. $[r,s] = e$. Likewise we have 
$[r,rs] = r(rs)r(rs)\inv = r^2 (rs)(rs)\inv = e$ and 
$[s, rs] = s (rs) s (rs)\inv = s^2 (rs) (rs)\inv = e$\footnote{
I later realized that computing $[g,h]$ is enough to determine
$[h,g]$. In particular if $[g,h] = x$ for some $x \in G$ then we have
$ghg\inv h\inv = x \implies x\inv = h g h\inv g\inv = [h,g]$. Alas,
there is some redundent calculation above.
}. Hence, $D_2$ is
also abelian. \wg{Question (feel free to ignore the following): was it sufficient to show that $r,s$
commute to show that $D_n$ is abelian? And generally speaking, if a
group is generated by $n$ elements $g_1, \cdots g_n$ which all
commute, is that sufficient to show that the group is abelian?}
\wg{(later), okay I have convinced myself. I think in general 
showing that all the generators of a finite group commute with each other is
enough to show that every commutator of a group is trivial. 
(even later) Moreso, I have now realized that if $[g,h] = e$ then we
also always have $[h,g] = e$.}

It follows that the
derived subgroup $G' = \Gamma^2 G$ is trivial for $G = D_1$ or $G =
D_2$.
Moreover, since $\Gamma^3 G$ is generated by the commutators $[g, e] = g e
g\inv e = e$ for each $g \in G$, it follows that $\Gamma^i G$ is
trivial for each $i > 1$ for both of these groups.

Now suppose $n \geq 3$ is odd. We show first that $\Gamma^2 G =
\langle r \rangle$ by computing all the commutators. Recall that
$\Gamma^2 G = G'$ is the subgroup generated by all commutators $[g,h]$
where $g \in G$ and $h \in \Gamma^1 G = G$. We need to compute the
following commutators: $[r^k,s], [r,r^ks], [s,r^ks]$ where $k = 1,
\cdots, n-1$. First recall the relation $rs = s r\inv$ and then
consider the following \[
	[r^k, s] = r^k s r^{-k} s = r^k r^{-k} s^2 = r^{-2k} \neq e
\]
where the last (in)equality follows from the fact that $n$ is odd.
In particular $k=1$ shows that $r^2 \in \Gamma^2 G$. 
Very similar calculations give the following
\begin{align*}
	[r, r^ks] &= r (r^k s) r\inv (s r^{-k}) = r^2 \neq e \\
	[s, r^ks] &= s(r^k s) s (s r^{-k}) = r^{-2k} \neq e
\end{align*}
Then, since we have already shown that $r^2 \in \Gamma^2 G$ we have
that $\Gamma^2 G = \langle r^2 \rangle$. The situation is even better
than this because for $n \geq 3$ odd we have $\langle r^2 \rangle =
\langle r \rangle$ in $D_n$. We have $r^2 = r \cdot r \in \langle r \rangle$.
Moreover, consider there are $n$ distinct powers of $r$ in $\langle
r^2 \rangle$, we have \[
\langle r^2 \rangle = \{r^2, r^4, \cdots, r^{n + 1} = r, r^{n+3}, \cdots,
r^{2n-2}, e\},
\]
since, again, $n$ is odd. That is, we have $r \in \langle r^2 \rangle$
and so indeed we have \[
	\Gamma^2 G = \langle r^2 \rangle = \langle r \rangle.
\]

Next we show that when $\Gamma^{i-1} G = \langle r \rangle$  then we
must have $\Gamma^i G = \anglee r$ for $i = 3, 4, \dots$ 
Recall that $\Gamma^i G$ is generated by $[g,h]$ where $g \in G$ and
$h \in \Gamma^{i-1}G$. Since $\Gamma^{i-1}G$ is only generated by a
single element we only need to compute the following:
\begin{align*}
	[r^k, r] &= e\\
	[s, r] &= r^{-2}\\
	[r^ks, r] &= r^{-2},
\end{align*}
following similar calculations to above. That is $\Gamma^i G = \anglee
{r^2}$ and we still have $\anglee{r^2} = \anglee r$.
That is, we have shown $\Gamma^i G = \anglee r$ for $i \geq 2$. 

Now we consider the case when $n \geq 4$ is even. First notice that we
can split this case into two further cases --- either $n = 2^k$ for
some $k$ or $n = 2^km$ for some $k$ and some $m \geq 3$ odd. If $n$ is
not a power of $2$ then its prime factor decomposition is $2^k m$
where $m$ is the product of all of its odd prime factors.

Now first consider the case where $n = 2^km$. Our above calculation
shows that $\Gamma^{2}D_{2^km} = \anglee{r^2}$ but now $\anglee{r^2}
\neq \anglee r$ since $r^2$ has even order. In particular
$\anglee{r^2} = \{r^2, r^4, \cdots, r^{2\cdot(2^{k-1}m)} = e\}$. 
Now we compute $\Gamma^3 G$ via direct computation of the generators
$[g,h]$ where $g \in G$ and $h \in \Gamma^2G = \anglee{r^2}$.
Following the now usual strategy, we have 
\begin{align*}
	[r^k, r^2] &= e \\
	[s, r^2] &= sr^2 s r^{-2} = r^{-4} \\
	[r^ks, r^2] &= r^{-4}.
\end{align*}
That is, $\Gamma^3 G = \anglee{r^4}$. Essentially the same calculation
will give $\Gamma^i G = \anglee{r^{2^{(i-1)}}}$ for $2 \leq i$.
\wg{The book claims that this simplifies to $\anglee{r^{2^k}}$ when
$i \geq k+1$, but im having trouble seeing why. come back to this
later}.

Now suppose $n = 2^k$ for some $k$. The same computation as above
gives $\Gamma^i G = \anglee{r^{2^(i-1)}}$ for $i \geq 2$. However, now
when $i \geq k+1$ we have $r^{2^(i-1)} = r^{2^(k+\ell)} =
(r^{2^k})^\ell = e^\ell = e$. And so, when $i \geq k+1$ we have 
$\Gamma^i G = e$. This last fact also follows since when $i = k+1$ we
have $\Gamma^i G = \anglee{r^{2^k}} = e$. In question $1$ we showed
that $\Gamma^i G \subseteq \Gamma^{i-1}G$ and so it would then follow
that all $\Gamma^i G = e$ for all $i > k+1$.
That is, when $n = 2^k$ we have that $D_n$ is nilpotent, and in particular solvable. 

Now we consider the derived series for $D_n$. Recall that $(D_n)^{(1)}
= \Gamma^2 D_n$, and then $(D_n)^{(i)}$ is generated by $[g,h]$ for
each $g, h \in (D_n)^{(i-1)}$. For $n=1,2$ we showed above that $D_n$
is abelian. It follows that all commutators are trivial, i.e.,
$G^{(1)} = e$ and then $G^{(i)} = 1$ for all $i \geq 1$.
For $n \geq 3$ we showed above that the derived subgroup $G' = \anglee
x$ is a cyclic subgroup, where either $x = r$ or $x = r^{2k}$.
In particular $G'$ is abelian, and so it follows that all commutators
$[g,h]$ with $g, h \in G'$ are trivial. And so for $n \geq 3$ we have
$(D_n)^{(1)} = \anglee{x}$ and $(D_n)^{(i)} = e$ for $i > 1$, where
$x$ depends on the exact form of $n$, as discussed above.

\end{solution}

\newpage

%------------------------- Problem 3 -----------------------

\begin{problem}
	\includegraphics[scale=0.8]{3.png}
	\hfill
\end{problem}
\begin{solution}
\begin{itemize}
\item In question $5$ we find all the groups of order $28$ partly by
studying the possible semidirect products $C_7 \rtimes_\theta C_4$, these semidirect
products give an extension of the abelian group $C_4$ by the abelian
group $C_7$. However, we find examples where $C_7 \rtimes C_4$ is
non-abelian. Namely when $\theta: C_4 \to Aut(C_7) \cong C_6$ is the
map of order $2$ given by $\theta(h) = (n \mapsto -n \equiv (7 -n))$
for each $h$.

\item \wg{big apologies. These are interesting though, I would be
curious whether the rest are true.}
\end{itemize}
\end{solution}

\newpage

%------------------------- Problem 4 -----------------------

\begin{problem}
	\includegraphics[scale=0.8]{4.png}
	\hfill
\end{problem}

\begin{solution}
Firstly, we will show that $\rho$ is a well defined map $H \to
Out(N)$. Let $h \in H$ and $\tih_1, \tih_2 \in G$ such that $\pi(\tih_1)
= \pi(\tih_2) = h$. We have 
$\rho(\tih_1) = f := (n \mapsto \tih_1 n \tih_1\inv)$ and
$\rho(\tih_2) = g := (n \mapsto \tih_2 n \tih_2\inv)$. 
Note that these are indeed automorphisms of $N$, as in the previous
homework we showed that conjugation by a fixed element is an
automorphism.
If we show that $\rho(\tih_1)$ and $\rho(\tih_2)$ lie in the same
coset of $Inn(N)$ then $\rho$ is well-defined. (Note: I believe this
map is not well defined as a map $H \to Aut(N)$). 

Recall that two elements $g,h$ of a group lie in the same coset of a normal
subgroup $N$ if $g\inv h \in N$. For our automorphisms $f,g$ we have
$g\inv = (n \mapsto \tih_2\inv n \tih_2)$. And so we have 
$(g\inv \circ f)(n) = \tih_2\inv \tih_1 n \tih_1\inv \tih_2$. Recall
that $N \normal G$ and so is closed under conjugation by definition.
In particular then $\tih_1 n \tih_1\inv \in N$ and $\tih_2\inv(\tih_1
n \tih_1 \inv) \tih_2 \in N$ since $\tih_1, \tih_2 \in G$. 
Thus $f,g$ have the same image in $Out(N)$ and so $\rho$ is well
defined with respect to the choice of $\tih$. 

Next we show that $\rho$ is a group homomorphism. Let $h_1, h_2 \in H$
and $\tih_1, \tih_2 \in G$ such that $\pi(\tih_1) = h_1$ and
$\pi(\tih_2) = h_2$. Moreover, since $\pi$ is a group homomorphism we
have $\pi(\tih_1\tih_2) = \tih_1 \tih_2$. Following a similar,
calculation to last week's homework, consider the following

\begin{align*}
	\rho(h_1h_2) &= \gamma_{\tih_1\tih_2} \\
		&= (n \mapsto \tih_1\tih_2 n (\tih_1\tih_2)\inv) \\
		&= (n \mapsto \tih_1 \tih_2 n \tih_2\inv \tih_1\inv) \\
		&= \gamma_{\tih_1} \circ \gamma_{\tih_2} \\
		&= \rho(h_1)\rho(h_2).
\end{align*}
Thus, the given $\rho$ is indeed a group homomorphism.

Now suppose $G = N \rtimes_\theta H$. We can state more precisely the
outer automorphism given by $\rho$. Let $h \in H$ and then all lifts
are of the form $\tih = (m, h)$ for some $m \in N$. Then, being
explicit about the details of the semidirect product, our map
$\rho(h) : \iota(N) \to \iota(N)$ acts as follows 
\begin{align*}
	\rho_h(n) &= (m, h) \cdot_\theta (n, e_H) \cdot_\theta (m, h)\inv \\
		&= (m,h) (n, e_H) (\theta_{h\inv}(m\inv), h\inv) \\
		&= (m \theta_h(n), h) (\theta_{h\inv}(m\inv), h\inv) \\
		&= (m \theta_h(n) (\theta_h \circ \theta_{h\inv}(m\inv), h h\inv) \\
		&= (m \theta_h(n) m\inv, e_H).
\end{align*}
Which induces the automorphism $f = (n \mapsto m\theta_h(n)m\inv) : N \to
N$. Note that $(\theta_{h} \theta_{h\inv}) = id_H$ since $\theta$ is a
group homomorphism $H \to Aut(N)$. 

We show that this is the same as the composition $H \to Aut(N) \to
Out(N)$. We have $h \mapsto \theta_h \mapsto \overline{\theta_h}$. 
Notice now that $\theta_h$ and $f$ are lie in the same coset of
$Inn(N)$. In particular \[
	\overline{\theta_h} = \overline{\gamma_m \theta_h} = \overline{f}
\]
since $\gamma_m = (n \mapsto m n m\inv)$ is one of the inner
automorphisms of $N$.
Hence, in the case where $G = N \rtimes_\theta H$ we have $\rho$ and
$H \to Aut(N) \to Out(N)$ give the same map.

One interpretation of this is that, whilst $\rho$ is a well defined
map $H \to Out(N)$, it is not a well defined map $H \to Aut(N)$.
However, in the case where $G$ is a semidirect product of $N$ and $H$
via $\theta$,
we have a preferred lift $h \mapsto (e_N, h) \in G$, 
and in fact there is a well defined map $H \to Aut(N)$, namely
$\theta$, whose projection gives the same map as $\rho$. 


\end{solution}

\newpage
%------------------------- Problem 5 -----------------------

\begin{problem}
    \includegraphics[scale=0.8]{5-1.png}
	\includegraphics[scale=0.8]{5-2.png}
	\hfill
\end{problem}

\begin{solution}
\begin{itemize}
\item Sylow's theorem I gives us that there exists a subgroup of order
$7$ in $G$, since $\abs{H} = 7^1 \cdot 4$ and $7$ does not divide $4$. 
Alternatively, Cauchy's theorem gives us that there exists an element
$g \in G$ with $\abs{g} = 7$, hence we have $\langle g \rangle
\leq G$. Moreover, Sylow III gives us that there's only a single Sylow
$7$ group. Indeed, if $n_7$ is the number of Sylow $7$ groups in
$G$ then Sylow III gives us that $n_7 \equiv 1 \mod 7$ and $n_7 \vert
4$. The only integer solving both these conditions is $n_p = 1$. 
Similarly, if we write $\abs G = 28 = 2^2 \cdot 7$ and notice that $2$
does not divide $7$, then Sylow I gives us that there exists a subgroup of
order $2^2 = 4$.

Let $N$ denote the $7$-subgroup of $G$, we argue that $N$ is normal. 
If $g \in G$ then recall $\gamma_g =
(\ell \mapsto g\ell g\inv)$ is an automorphism of G. Therefore $\abs{\gamma_g(N)} =
\abs{N}$. However, there's a unique subgroup of order $7$ in $G$ and
so the image $\gamma_g(N) = N$ for all $g \in G$. That is, $N$ is
closed under conjugation by elements in $G$ and so $N$ is normal by
definition. We have shown that $G$ has a normal subgroup of order $7$
and in fact we have found that $N \cong C_7$.


\item Recall \wg{or perhaps I shall prove} that $Aut(N) = Aut(C_7)
\cong C_6$. We determine the homomorphisms $C_4 \to Aut(C_7) \cong
C_6$. 
Consider $C_4$, once we have specified where a generator
$\sigma \in C_4$ is mapped to in $C_6$ then we have determined the
homomorphism $C_4 \to C_6$.
Since $\abs \sigma = 4$ we must have $\abs{\theta(\sigma)} = 4$ or
$\abs{\theta(\sigma)} = 2$, for $\theta$ non-trivial,
since the image of an element under a homomorphism must have order
which divides the original element's order.
Notice that there's only a single element of order $2$ in $C_6$, and
no elements of order $4$. And
so there's one trivial homomorphism $C_4 \to C_6$ and one non-trivial map $\overline \theta: C_4
\to N$.
Since $\overline\theta(\sigma)$ has order two we can deduce that it is the
automorphism which sends each element of $C_7$ to its inverse. That is 
$\overline\theta(\sigma) = (n \mapsto 7 - n)$. And, of course, the
trivial map $\theta_{\text{triv}}(\sigma) = (n \mapsto n)$ for each
$\sigma \in C_4$. 

We use similar reasoning to determine the maps $\theta: C_2 \times C_2
\to Aut(N) \cong C_6$. One generating set of $C_2 \times C_2$ is
$\{(0,1), (1,0)\}$ and again, once we determine where these elements
are mapped to by $\theta$ we have determine the entire homomorphism
$\theta: C_2 \times C_2 \to C_6$. Now each generating element has
order two, and so any non-trivial $\theta$ maps both the generating
elements to the unique element of order $2$ in $C_6$. 
And so, again, we have one trivial map $\theta_{\text{triv}}: C_2
\times C_2 \to C_6$ and one non-trivial map $\tilde \theta: C_2 \times
C_2 \to C_6$. The automorphisms $\tilde \theta((0,1)) =
\tilde\theta(1,0)$ are both the same as the one described above ---
$(n \mapsto 7 - n \equiv -n)$.

\item Determining all the possible semi-direct products $C_7 \rtimes
H$ with $H = C_4$ or $H = C_2 \times C_2$ will tell us the possible
group laws on $G$. Notice that $N \cap H = \{e \}$ for $H = C_4$ or
$C_2 \times C_2$. This follows since every non-identity element of $N \cong C_7$ is
order $7$, meanwhile there are no elements of order
$7$ in either $C_4$ or $C_2 \times C_2$. 
\wg{We also need to show that $NH = G$}. 
Then it follows that $G \cong N \rtimes_\theta H$ for $H = C_4$ or $H
= C_2 \times C_2$ and one of the $\theta$ described above.

With all possible homomorphisms $H \to Aut(N)$ described above,
we can determine all the semi-direct products $N \rtimes H$.
First suppose $H = C_4$ and $\theta : C_4 \to C_6$ the trivial map.
That is $\theta(h) = (n \mapsto n)$ for each $h \in H$. We have the
following group product for $N \rtimes_\theta H$:
\begin{align*}
	(n_1, h_1) \cdot_\theta (n_2, h_2) 
		&= (n_1 \theta_{h_1}(n_2), h_1 h_2) \\
		&= (n_1 n_2, h_1 h_2).
\end{align*}
That is, then $N \rtimes_\theta H$ is isomorphic to $C_7 \times C_4
\cong G$.
The same calculation will give us that when $H = C_2 \times C_2$ and
$\theta : C_2 \times C_2 \to Aut(N)$ is the trivial map, we also have
$G \cong C_7 \times C_2 \times C_2$. 

Now we determine the products given by the non-trivial $H \to Aut(N)$.
Note that when $H = C_4$ and $\theta(h) = (n \mapsto -n)$ we have a
non-commutative group structure on $C_7 \rtimes_\theta C_4$. Consider
\begin{align*}
	(n_1, h_1) \cdot_\theta (n_2, h_2) &= (n_1-n_2, h_1 h_2)
	\intertext{Meanwhile,}
	(n_2, h_2) \cdot_\theta (n_1, h_1) &= (n_2-n_1, h_1 h_2) \neq (n_1-n_2, h_1h_2),
\end{align*}
where we take $C_7$ as a group under addition and $C_4$ as a group
whose product is written as ``multiplication''. Similar reasoning I
think shows us that when $H = C_2 \times C_2$ and $\theta(h) = (n
\mapsto -n)$ we also have $C_7 \rtimes_\theta C-2 \times C_2$ as
non-abelian. These groups may be isomorphic to something nice, I'm not
sure, I'd have to stare at them a bit longer.

In any case, we have found every possible group structure on $G$, a
group of order $28$. There are two abelian group structures, and two
non-abelian group structures.

\end{itemize}
\end{solution}

\newpage
%------------------------- Problem 4 -----------------------

\begin{problem}
	\includegraphics[scale=0.8]{6.png}
	\hfill
\end{problem}

\begin{solution}
Aluffi points us to an example in the text, and so I will reprove that
example. We prove that if $\abs{G} = mp$ for some prime $p$ and some
$m < p$ then $G$ is not simple. \wg{Quick question: The text seems to require $m > 1$,
however, I cannot see where that condition is needed in the proof.
(Maybe in the hypothesis for the Sylow Theorems?)}
Recall Sylow III gives us that there are $1\mod p$ subgroups of order
$p$ in $G$. We show that there is in fact only a single $p$-subgroup
in $G$.

Suppose that there is more than one $p$-subgroup of order
$p$, then there must be at least $p+1$ $p$-subgroups.
Notice that if $H,H' \leq G$ are distinct $p$-subgroups then they can
only intersect at the identity. Indeed, suppose $g \in H \cap H'$.
Then, since $H,H'$ are groups, we have in particular $\anglee g \leq
H$. However, by Lagrange's theorem, we then have $\abs{g}$ divides $\abs
H$. However $H$ is of prime order, and so $\abs{g} = 1$ or $\abs g =
p$. If $\abs g = p$ then we in fact have $\anglee g = H$. However, 
we'd then also have $\anglee g = H'$ which contradicts the assumption that
$H$ and $H'$ are distinct. And so we must have $\abs g = 1$, in other
words, $g = e$. Thus, $H,H'$ may only intersect at the identity. And,
in fact, they always intersect at the identity, since they are both
subgroups.

Now, we have at least $p+1$ distinct $p$-subgroups, each of these
subgroups must contain $p-1$ elements which are not contained in any
of the other $p$-subgroups. That is, we have $1 + (p+1)(p-1) = p^2$
distinct elements of $G$ contained across all of these subgroups. However, $p^2 > mp =
\abs G$ and so we have a contradiction. That is, there must be exactly
one $p$-subgroup in $G$.

Let us denote the unique subgroup of order $p$ by $N$.
We show $N$ is normal in $G$. Let $g \in
G$ and consider the automorphism $\gamma_g = (n \mapsto
g n g\inv)$. Since $\gamma_g$ is an automorphism, we must have that
the image $\gamma_g(N)$ is a subgroup and must be of order $p$. 
However $N$ is the unique subgroup of order $p$ and so we actually have $\gamma_g(N) =
N$. This holds for each conjugation automorphism $\gamma_g$ with $g
\in G$. In other words, $N$ is normal by definition. We have found a
normal subgroup of $G$, and so $G$ is not simple.

Finally, notice that $\abs G = 57 = 3 \cdot 19$, then above lemma
gives that any group of order $57$ cannot be simple.
\end{solution}

\newpage

%------------------------- Problem 4 -----------------------

\begin{problem}
	\includegraphics[scale=0.8]{7.png}
	\hfill
\end{problem}

\begin{solution}
Suppose $G$ is simple, and let $p$ be a prime which divides $\abs G$. 
Notice that we have $\abs G = p^k \left( \Pi_i p_i^{\alpha_i}
\right)$, by definition of $\abs G$'s prime factorization. In
particular $p$ does not divide $m := \Pi_i p_i^{\alpha_i}$.
And so, we are free to apply the Sylow theorems in the following
manner.

Recall Sylow I and Sylow III gives us that $0 < n_p \equiv 1\mod p$
and $n_p \vert m$. In particular, if $n_p = 1$ then the unique
$p$-Sylow subgroup would be normal (see arguments in question $6$, for
example), however $G$ is simple and so we must have $n_p > 1$. 

Now, let $N_p$ be a set containing each of the $p$-Sylow subgroups,
that is \newline $N_p := \left\{ H \leq G : \abs{H} = p^k \right\}$. Allow $G$
to act on $N_p$ by conjugation $g \circlearrowright H = g H g\inv$. 
Recall that $G$ acting on itself by conjugation is an automorphism,
and so in particular $\abs{g H g\inv} = \abs{H}$. In other words the
conjugation of a $p$-Sylow subgroup is again a $p$-Sylow subgroup and
so our proposed action is well-defined.

This action induces a homomorphism to the symmetric group on $n_p$
elements $\phi: G \to S_{n_p}$ defined as follows.
If we label the elements in $N_p$ by
integers in $[n_p]$, denote the integer associated with $H \in N_p$
with $ind(H)$, and denote the $p$-Sylow subgroup associated with the
integer $i$ as $H(i)$,
then $\phi(g)$ is the permutation where $i \mapsto ind(g H(i) g\inv)$.
Recall that $G$ acting on itself is an automorphism, and in particular
our action on $N_p$ is a bijection, hence $\phi(g)$ with this
definition is a well-defined permutation of $[n_p]$. \wg{The logic of
this last sentence is a bit iffy in my brain, does this sound correct?}

By the first isomorphism theorem we have $\im(\phi) \leq S_{n_p}$ and
in particular, by Lagrange's theorem, $\abs{\im(\phi)}$ divides
$\abs{S_{n_p}} = n_p!$. Moreover, $\im \phi \cong G / \ker \phi$, and
so another application of Lagrange's theorem gives \[
	\frac{\abs G}{\abs{\ker \phi}} \big\vert \abs{S_{n_p}} = n_p!
\]
Next, we show that $\ker \phi$ is trivial. Notice that $\ker \phi$ is
a normal subgroup of $G$. Let $h \in \ker \phi$ meaning $h H h\inv =
H$ for each $p$-Sylow subgroup in $N_p$. Now let $g \in G$, we have \[
	(g h g\inv) H (g\inv h\inv g) = g (h H' h\inv) g\inv
		= g H' g \inv = g (g\inv H g ) g\inv = H,
\]
Where $H' \in N_p$. 
That is $g h g\inv \in \ker \phi$ and so $\ker \phi \normal G$ by
definition. Recall that $G$ is simple, and so $\ker \phi$
must either be trivial or all of $G$. However, recall that Sylow II
gives that if $H, H' \in N_p$ then $H,H'$ are conjugates of each
other: $H = g H' g\inv$ for some $g \in G$. Moreover, earlier we
showed that $\abs{N_p} = n_p > 1$. That is, there must be some
non-trivial element of $g$ which does not fix all of $N_p$. In other
words, $\ker \phi$ is a proper subgroup of $G$. It follows then that
$\ker \phi$ is trivial and $\abs{\ker \phi} = 1$. And so, we have \[
\abs{G} = \frac{\abs G}{\abs{\ker \phi}} \big\vert \abs{S_{n_p}} =
n_p!,
\]
as desired.



\end{solution}

\newpage

%------------------------- Problem 4 -----------------------

\begin{problem}
	\includegraphics[scale=0.8]{8.png}
	\hfill
\end{problem}

\begin{solution}
\begin{enumerate}[(a)]
\item Let $\phi$ denote the suggested map $\phi(a_i) = (i, i+1)$. We
take the suggested map and extend it so that it's a homomorphism,
i.e., $\phi(a_i \cdot a_j) = (j,j+1)(i,i+1)$ (note, here my elements
of $S_n$ act on the right of permutations of $[n]$, I do this so that
my notation for transposition decomposition is correct later). 
Note that there are exactly $n-1$ transpositions of the form $(i,i+1)$
in $S_n$, namely $(1,2), (2,3), \cdots, (n-1, n)$. It then follows
that $\Sigma_n$ bijects onto the set of $(i,i+1)$-transpositions in
$S_n$. 

We argue that this
homomorphism is indeed surjective.
Recall that any
element $\sigma \in S_n$ has a disjoint cycle decomposition. That is we can
always write $\sigma = ((a_1)_1, (a_1)_2, \cdots, (a_1)_{r_1}) \cdots ((a_k)_1,
\cdots, (a_k)_{r_k})$ (apologies for this notation). 
If $(a_1, \cdots, a_k)$ is a cycle in $S_n$
then we have \[
	(a_1, \cdots, a_k) = (a_1, a_2)(a_1, a_3) \cdots (a_1, a_k)
\]
Now, notice that $(a_1, a_2)$ may not be of the form $(i, i+1)$, we
could have, say, $(a_1, a_2) = (1, 4)$ if $n \geq 4$. However, we can
decompose these transpositions further. Suppose $(a_1, a_2)$ is a transposition in
$S_n$ with $a_1 < a_2$, then we have 
\begin{align*}
(a_1, a_2) = &[(a_1, a_1 + 1) (a_1+1, a_1 + 2) \cdots (a_2 -2, a_2 -1)] \\
	& \cdot (a_2 - 1, a_2) \cdot \\
	& \cdot [(a_2 - 2, a_2 -1) \cdots (a_1 + 1, a_1 + 2) (a_1, a_1+1)], 
\end{align*}
where, recall, our transpositions act on the right of a given
permutation. Note that since each transposition is order two, this is
really a conjugation of $(a_2 - 1, a_2)$ by the element $(a_1, a_1 +
1) \cdots (a_2 - 2, a_2 - 1)$. The equality above is probably easiest to see with an
example. Suppose $n = 4$ and notice that indeed \[
(1, 4) = (1,2) (2,3) (3,4) (2,3) (1,2) = [(1,2)(2,3)](3,4)[(1,2)(2,3)]\inv.
\]

In any case, we've recalled that every element of $S_n$ has a disjoint
cycle decomposition, we've shown that every cycle has a transposition decomposition,
and every transposition has a $(i,i+1)$-decomposition. It follows then
that every element of $S_n$ has a $(i,i+1)$-transposition
decomposition. Thus $\phi$ is surjective since it surjects onto the
set of $(i,i+1)$ transpositions in $S_n$.

\item 
First suppose $\abs{i -j} \neq 1$ and without loss of generality
suppose $0 \leq i+1 < j \leq n-1$. Then consider 
\[
	a_i a_j = a_i (a_ia_j)^2 a_j = (a_i)^2 a_j a_i (a_j)^2 = a_j a_i,
\]
using the given relations. In other words, each such $a_i, a_j$
commute. Using a similar idea, consider 
\begin{align*}
	a_{i+1}a_i a_{i+1} 
		&= a_{i+1} a_i (a_ia_{i+1})^3 a_{i+1} \\
		&= a_{i+1} a_i (a_ia_{i+1}) (a_ia_{i+1})(a_ia_{i+1}) a_{i+1} \\
		&= a_{i+1} (a_i)^2 a_{i+1} a_i a_{i+1} a_i (a_{i+1})^2  \\
	 	&= (a_{i+1})^2 a_{i+1}a_i a_{i+1} \\
	 	&= a_{i+1}a_i a_{i+1}, 
\end{align*}
holds for each $1 \leq i \leq n-2$. 

\item
If $w$ is a word in $\Sigma_n$ which does not contain $a_{n-1}$ then we are
trivially done. 
Now, let $w$ be a word in $\Sigma_n$ which contains $a_{n-1}$ but no
instances of the letter $a_{n-2}$. That is $w$ is a word such that
$\abs{n-1 - j} > 1$ for all $a_j \in w$ (where we use the notation
$a_j \in w$ to mean ``$w$ contains the letter $a_j$''). Note that the
given relations imply that $(a_i)\inv = a_i$ for all $i \in [n]$, in
particular, we do not have $a_j = (a_{n-1})\inv$ for some $j < n-1$.
That is $\Sigma_{n-1} \leq \Sigma_n$.
Since $a_{n-2} \not\in w$ we have that $a_{n-1}$ commutes with every
letter in $w$ and so we can write $w = a_n w'$ where $w' \in
\Sigma_{n-1}$.

Now suppose $a_{n-2}, a_{n-1} \in w$ with $a_{n-2} <_w a_{n-1}$
(meaning, $a_{n-2}$ is ``to the left of'' $a_{n-1}$ in $w$), but
$a_{n-3} \not\in w$. That is $w = \bar a_1 \bar a_2 \cdots a_{n-2}
\cdots a_{n-1} \cdots \bar a_k$ where each $a_{n-3} \neq \bar a_i \in
\Sigma_{n-1}$. Now $a_{n-1}$ commutes with everything to its left
until ``it hits'' $a_{n-2}$. That is $w = \bar a_1 \cdots a_{n-2}
a_{n-1} \cdots \bar a_k$. And, $a_{n-1}$ does not commute with
$a_{n-2}$, however, we can ``push them down the word together'', that
is, notice $w = \cdots \bar a_\ell a_{n-2}a_{n-1} \cdots = \cdots a_{n-2}
\bar a_\ell a_{n-1} \cdots = \cdots a_{n-2}a_{n-2} \bar a_\ell
\cdots$, Since $\ell < n-2 < n-1$. And so it follows that $w =
a_{n-2}a_{n-1} w$ where $w \in \Sigma_{n-1}$.

The same logic above applies to any word of the form\footnote{
	Note that we only care about the existence of letters $w_{k-1}$ to
	the left
	relative to $w_{k}$, since we are attempting to ``push the letters
	to the left of the word''. I.e. our argument still holds if there
	are letters $a_{i-1} >_w a_{i}$, even if we do not explicitly
	cover it.
} 
	$w = \cdots w_{n - k} \cdots w_{n -(k+1)} \cdots w_{n-1} \cdots $
	(note, although i was too lazy to write it as such, $w$ is a finite
	word). 
	That is, the logic above applies so that we can ``push down
	$a_{n-1}$ until it hits $a_{n-2}$, and then push the block
	$a_{n-2}a_{n-1}$ until they hit $a_{n-3}$, etc, until the block
	$a_{n-k}\cdots a_{n-1}$ is at the left of the word''. More 
	precisely, we 
	can write
\begin{align*}
	w 
	&= \cdots w_{n - k} \cdots w_{n-(k+1)}\bar a \cdots \bar a w_{n-2} \cdots w_{n-1} \cdots && \text{where $\bar a \in \Sigma_{n-1}$ possibly distinct} \\
	&= \cdots w_{n - k} \cdots w_{n-(k+1)}\bar a \cdots \bar a w_{n-2} w_{n-1} \cdots \\
	&= \cdots w_{n - k} \cdots w_{n-(k+1)} \cdots w_{n-2} w_{n-1} \bar a \cdots \\
	&= \cdots w_{n - k} w_{n-(k+1)} \cdots w_{n-2} w_{n-1} \cdots \\
	&= w_{n - k} w_{n-(k+1)} \cdots w_{n-2} w_{n-1} \cdots \\
	&= w_{n - k} w_{n-(k+1)} \cdots w_{n-2} w_{n-1} w' && w' \in \Sigma_{n-1},
\end{align*}
for $1 \leq k \leq n-1$, as sought. (Apologies for the cumbersome notation.)

\item 
Following the definition in the question $\Sigma_{n-1}$ is the group
generated by $n-2$ elements $\tilde a_1, \cdots, \tilde a_{n-2}$
satisfying the relations $(\tilde a_i)^2 = (\tilde a_i \tilde a_j)^2 =
(\tilde a_i \tilde a_{i+1})^3 = 1$ for appropriate indices. Define a
map $\Sigma_{n-1} \to \Sigma_n$ by $\tilde a_i \mapsto a_i$ for each
$i$, and extend this so that it's a homomorphism, i.e., $(\tilde a_i
\tilde a_j) \mapsto a_i a_j$. Then the relations of $\Sigma_{n-1}$ are
satisfied by the relations in $\Sigma_{n}$ by definition. Hence the
image of this map is the subgroup generated by $a_i$ for $i = 1,
\cdots, n-2$. Moreover, this map is injective.

We then have a chain of inclusions \[
	\Sigma_1 = 1 \hookrightarrow \Sigma_2 \hookrightarrow \cdots
	\hookrightarrow \Sigma_{n-1} \hookrightarrow \Sigma_{n}.
\]
Now consider, $\abs{\Sigma_2} = 2$ by definition. The previous
	part shows that the words in $\Sigma_3$ are of one of the
	following forms $e \cdot
	w', a_1 \cdot w', a_1 a_2 \cdot w'$, where $w' \in \Sigma_2$.
	Therefore, there are at most $3 \cdot \abs{\Sigma_2} = 3 \cdot 2 = 6$
	words in $\Sigma_3$. 
	Generally, the previous part shows that there are at most $k
	\abs{\Sigma_{k-1}}$ words in $\Sigma_k$. In particular, unpacking
	the recursion, \[
		\abs{\Sigma_n} \leq n \abs{\Sigma_{n-1}} \leq n!
	\]

\item In part $(a)$ we showed that $\phi$ is a homomorphism and that
it surjects onto $S_n$, which has $n!$ elements. It follows then that
$\abs{\Sigma_n} \geq n!$. Combined with the statement in the previous
part, it follows that in fact $\abs{\Sigma_n} = n!$. And so $\phi$ is
in fact an isomorphism. I.e. $\Sigma_n \cong S_n$, and the description
of $\Sigma_n$ given in the question is then actually a presentation
for $S_n$. 

\end{enumerate}
\end{solution}

\newpage



\end{document}
