\documentclass[12pt,letterpaper,boxed]{hmcpset}
\usepackage[margin=1in,headheight=14pt]{geometry}
\usepackage{amsfonts, amsmath, amssymb, enumerate, fancyhdr, gensymb, lastpage, mathtools, parskip, graphicx}
\usepackage{xcolor, tikz-cd}
\newcommand{\wg}[1]{\textcolor{violet}{#1}}
\newcommand{\OO}{\mathcal O}
\newcommand{\Q}{\mathbb Q}
\newcommand{\R}{\mathbb R}
\newcommand{\C}{\mathcal C}
\newcommand{\Z}{\mathbb Z}
\newcommand{\abs}[1]{\left|#1\right|}
\newcommand{\im}{\text{im }}
\newcommand{\inv}{^{-1}}
\newcommand{\normal}{\unlhd} %% one can also use \trianglelelefteq
\newcommand{\anglee}[1]{\langle #1 \rangle}
\newcommand{\F}{\mathbb F}
\usepackage[shortlabels]{enumitem}

% Numbering macros
\pagestyle{fancy}
\lhead{Will Gilroy}
\chead{Algs Homework \#}
\rhead{03 November 2021}
\lfoot{}
\cfoot{}
\rfoot{Page\ \thepage\ of\ \pageref{LastPage}}

\linespread{1.5}

\newcommand\blankpage{
    \thispagestyle{empty}
    \addtocounter{page}{-1}
    \newpage}
\renewcommand\footrulewidth{0.4pt}

\begin{document}

\problemlist{Algorithms HW } 

%------------------------- Problem 1 -----------------------

\begin{problem}[1]
	\hfill
\end{problem}

\begin{solution}
\end{solution}

\newpage

%------------------------- Problem 2 -----------------------

\begin{problem}
	\includegraphics[scale=1]{2.png}
	\hfill
\end{problem}

\begin{solution}
I will outline the general procedure for how we decompose $M$ a finitely
generated module over a PID $R$ into it's direct sum of cyclic modules.

Since $M$ is finitely generated, say by $n$ generators, then we have a
surjection from the free module $R^n$ into $M$, $f: R^n \twoheadrightarrow M$. Moreover,
$\ker(f)$ is also finitely generated as a submodule of a finitely
generated module over a Noetherian ring (since $R$ is a PID). 
Let $m := rank \ker(f)$ and then,
by similar reasoning, we have another map given by the composition $g: R^m \twoheadrightarrow
\ker f \hookrightarrow R^n$. Moreover, by the first isomorphism
theorem of modules, we have that $M \cong R^n/\im(g) =
\text{coker}(g)$. And so, if we determine the $\text{coker}(g)$ we
have a representation of $M$.

Since $g: R^m \to R^n$ we can represent it by a $m \times n$ matrix
$A$.
Then, if we put $A$ into Smith Normal Form (SNF) (which amounts to
representing the same transformation under a change of basis of $R^m$
and $R^n$) then we can write $M \cong \anglee{e_1, \cdots, e_n} /
\anglee{d_1e_1, \cdots, d_k e_k}$ where $d_i$ are the Smith Normal
Form entries, and where $k \leq n$. 

With this procedure outlined, let me now actually answer the given
questions lol.

\begin{itemize}
\item We have $M = \Z^2/\anglee{(18, 30)}$ and the surjection $f: \Z^2 
\twoheadrightarrow M$ via the quotient map. Moreover, manifestly, we
have $\ker(f) = \anglee{(18,30)}$ and so we have a map $g: \Z \to
\Z^2$ via $g(a) = a\cdot (18,30)$. We can represent $g$ as the $2
\times 1$ matrix $A = [18, 30]^T$. Let us now put $A$ into Smith
Normal Form. The SNF of $A$ is of the form $[d_1, 0]^T$ and we have
that generally the first Smith factor $d_1$ is the greatest common
divisor of all the entires of $A$. That is $A \sim [gcd(18, 30), 0]^T
= [6, 0]^T$. And now we can write our decomposition \[
M \cong \frac{\anglee{e_1, e_2}}{\anglee{6e_1}} \cong \Z/6\Z \oplus
\Z.
\]

\item Following in a similar fashion to part $(a)$, we have a
surjective map $f: \Z[i]^3 \hookrightarrow M$ given by the quotient map.
We have $\ker f \cong \anglee{(2 + 2i, 8 + 6i, 6), (1+i, 7+3i,
3-3i)}$ and then the matrix representing $g: \Z[i]^2 \to \Z[i]^3$ is
given by \[
	A = \begin{bmatrix}
		2 + 2i & 1 + i \\
		8 + 6i & 7 + 3i \\
		6 & 3 - 3i
	\end{bmatrix}.
\]
Whose SNF will be of the form \[
A \sim \begin{bmatrix}
	d_1 & 0 \\
	0 & d_2 \\
	0 & 0 	
\end{bmatrix}.
\]

Once, again we can compute $d_1 = gcd(1 + i, 2 + 2i, 8+6i, 7+3i, 6,
3-3i)$. Notice that $1+i$ is a Gaussian Prime (it has norm $2$, and so
it is straightforward to verify by enumeration of elements with smaller norm 
that it has no divisors other than $1$ and itself). And so, if $1+i$ divides
every element in $A$ then $d_1 = 1+i$ otherwise $d_1 = 1$.

It turns out that $1+i$ divides every element in $A$. I will only
outline how I attempted to divides one element by $1+i$, because I am
curious if there's a better method. But I will not write out the
details of every check. 
Consider $7+3i$, we want to know if there's some Gaussian integer $a$
such that $a (1+i) = 7+3i$. Notice that $\| 1+ i \| = 2$ and $\| 7 +
3i \| = 58$. And so any such $a = x + yi$ must satisfy $\| a \| = 29$. 
Enumerating all the squares up to $100$ gives us that we must have
$\|x\| = 4$ and $\|y\| = 25$ or vice versa.
Then, checking the four possibilities for $x,y$ gives $7 + 3i =
(1+i)\cdot(5-2i)$. And so, $1+i$ is a divisor of $7+3i$. 
Using a similar method gives that $1+i$ is a divisor of every element
in $A$ and so $d_1 = 1+i$. 
\wg{oh, shit, what if there's a greater common divisor. nooo, since
$1+i$ is a Gaussian prime, the gcd of the whole list is already
``bounded above'' by this number.}

Now to compute $d_2$ we have that the $2$nd invariant factor of $A$,
given by the gcd of all the $2\times 2$ minors of $A$, is equal to
$d_1 d_2$. Computing all the $2 \times 2$ minors of $A$ gives 
\[
	d_2 = \text{gcd}(-24i, 6-6i, 6+6i).
\]
Given the computations I already did for $d_1$, it is easy to write
down a unique (up to units) factorization of each of the elements
\begin{align*}
	6+6i &= 6(1+i) = 3 (1+i)^2 (1-i) \\
	6-6i &= 6(1-i) = 3 (1-i)^2 (1+i) \\
	-24i &= -12(1+i)^2,
\end{align*}
And then we can inspect that $d_2 = 1+i$. \wg{Quick question, 
I noticed that we can also write $-24i = 12 (1-i)(-1+i)$, which would
then imply that the gcd of these elements is $(1-i)$. Although, of
course, $1+i = i(1 - i)$, and so is the gcd only unique up to units?
(I suppose even in $\Z$ it is true that the gcd is unique only up to
$\pm 1$.)}

To summarize, the SNF of $A$ is given by \[
A \sim \begin{bmatrix}
	1 + i & 0 \\
	0 & 1+i \\
	0 & 0 .
\end{bmatrix}
\]
And hence, our decomposition of $M$ is given by \[
M \cong \frac{\anglee{e_1, e_2, e_3}}{ \anglee{(1+i)e_1, (1+i)e_2}} 
\cong \Z[i] / (1+i) \oplus \Z[i] / (1 + i) \oplus \Z[i].
\]
\wg{Question for self, how can we tell that each of these factors is
in fact cyclic? Is $\Z[i]/(1+i) \cong \Z$?}

\wg{come back and do part $3$ later}


\end{itemize}

\end{solution}

\newpage

%------------------------- Problem 3 -----------------------

\begin{problem}
	\includegraphics[scale=0.8]{3.png}
	\hfill
\end{problem}
\begin{solution}
Let $M \leq R^n$ be an $R$-submodule of $R^n$.
Recall that submodules of finitely generated modules over Noetherian
rings are again finitely generated. And so $M$ is finitely generated,
with generators $x_1, x_2, \cdots, x_k$, say.
Moreover $M$, as a finitely generated module over a PID can then be
decomposed into a direct product of cyclic modules.
\wg{why does it matter then if we can put $A$ into SNF?}
\wg{If $M$ is a direct sum of cyclic submodules, does that mean it's
already free? What if each of the factors has more than one generator?
no, because they are cyclic. What if they are not isomorphic to R?
Ah, in general, they will not be.
}

Since $M$ is finitley genrated, we have a surjective map $R$-linear
map $R^k \twoheadrightarrow M$ via $e_i \mapsto x_i$. 
Moreover, we have a composition \[
	R^k \twoheadrightarrow^{\sigma} M \twoheadrightarrow^{\iota} R^n,
\]
which can be represented in terms of a $k \times n$ matrix $A$. 
We claim that $M \cong \im(A)$. 
First note that since $\iota$ is an injection, we have $M \cong
\im(\iota)$. Now, since $\sigma$ is a surjection we have, for all $a
\in R^k$ $(\iota \circ \sigma)(a) = \iota(m) = m \in R^n$.
That is, we can identify $(\iota \circ \sigma)(a)$
with an element of $m$, for all $a \in R^k$.

Notice that if $A \sim A'$ then $im(A) \cong im(A')$, since $A$ and
$A'$ represent the same $R$-linear transformation under some change of
basis on $R^k$ and $R^m$. In particular we can put $A$ into Smith
Normal Form \[
	A \sim \begin{bmatrix}
		d_1 & & & & \\
		& d_2 & & & \\
		& & \cdots & & \\
		& & & d_{\text{min}(k,n)} & 
	\end{bmatrix}
\]


\wg{wait, im confused, does this not just amount to the idea
that we can decompose $M$ into direct sum of cyclic modules.
perhaps a proposition for a basis is each of the generators of the
cyclic factors? are these necessarily $R$-linearly independent? }


\end{solution}

\newpage

%------------------------- Problem 4 -----------------------

\begin{problem}
	\includegraphics[scale=0.8]{4.png}
	\hfill
\end{problem}

\begin{solution}
Suppose we have two matrices $A,B$ over a field (so that $k[x]$ is a
PID) have the same characteristic and minimal polynomials.
To show that $A \sim B$ we can show that they have the same rational
canonical forms.
Recall that the rational canonical form of a matrix is of the form \[
	\begin{bmatrix}
		C_{f_1} & & \\
		& C_{f_2} & \\
		& & C_{f_3} 
	\end{bmatrix},
\]
	where the $C_{f_i}$ are companion matrices to the $f_i$. And the
	$f_i$ are the invariant factors for the cyclic decomposition of
	the finitely generated $k[x]$-module induced by the $k$-linear
	transformation $A$, say.
	That is, we can show that $A \sim B$ if we can show that they have
	the same invariant polynomials.
	In generality, since $A$ and $B$ are $3 \times 3$ matrices, let us
	say that their characteristic polynomials are of degree $3$.
	Let $f_1, f_2, f_3$ be the invariant factors for $A$ and let $g_1,
	g_2, g_3$ be the invariant factors for $B$, where $f_1 \vert f_2
	\vert f_3$ and $g_1 \vert g_2 \vert g_3$. 

Recall from Aluffi that the characteristic and minimal polynomials are
related to the invariant factors by $P_A = f_1f_2f_3$ and, since the
minimal polynomial is defined to be the minimal degree polynomial
dividing $P_A$, $m_A = f_1$.

Since $m_A = m_B$ we have $f_1 = g_1 := k$. Since $k$ is in particular an
integral domain we then have $k(f_2f_3 - g_2g_3) = 0$, and since $k
\neq 0$, $f_2f_3 = g_2g_3$.
\wg{Now how, from here, do we show that $f_2 = g_2$ and $f_3 = g_3$?}
\wg{probably also need to discuss the cases of the different degrees
also}

\wg{The following is the if direction.}
On the other hand, suppose that $A \sim B$. Then $A = C B C\inv$ for
some invertable matrix $C \in GL_3(k)$. 
Note that conjugation by an invertible matrix amounts to a change of
basis of the original transformation. Then $C B C\inv$ has the same
eigenvalues as $B$ counted with multiplicity (although, it's eigenvectors must change under this
change of basis). And so, since the characteristic polynomial of a
linear transformation is determined by its eigenvalues with
multiplicity, $A = C B C\inv$ has the same characteristic polynomial
as $B$.

Now we cliam that $A$ and $B$ also have the same minimal polynoimal. 
\wg{i think i might need to think about their jordan normal forms 
to understand this one}


\end{solution}

\newpage

%------------------------- Problem 5 -----------------------

\begin{problem}[4]
	\hfill
\end{problem}

\begin{solution}
\end{solution}

\newpage

%------------------------- Problem 6 -----------------------

\begin{problem}
\includegraphics[scale=0.7]{6.png}
\hfill
\end{problem}

\begin{solution}
Given the roots of a polynomial we genreally can write \[
f(x) = c(x - \alpha_1) \cdots (x - \alpha_n).
\]
However, since $f$ is monic, expanding the linear terms will give that
$c = 1$. 
To write the companion matrix of $f$ we need $f$ in the form $a_0 +
a_1 x + \cdots + a_{n-1}x^{n-1} + x^n$. 
Expanding the terms gives us \[
f = x^n + \binom{\alpha}{1}x^{n-1} + \binom{\alpha}{2}x^{n-1} +
\cdots + \binom{\alpha}{n-1}x + \binom{\alpha}{n},
\]
where $\binom{\alpha}{k}$ means sum all the $k$-subsets of $\left\{
\alpha_i \right\}$. I.e. $\binom{\alpha}{k} := \sum_{I \subseteq
\{\alpha_i\},\abs{I} = k} \Pi_{i \in I}\alpha_i$
Now we can write the companion matrix $C_f$ as 
\[
C_f = \begin{bmatrix}
	0 & 0 & 0 & \cdots & -\binom{\alpha}{n} \\
	1 & 0 & 0 & \cdots & -\binom{\alpha}{n-1} \\
	& & & \vdots & \\
	0 & \cdots & & 1 & -\binom{\alpha}{1}
\end{bmatrix}.
\]
From here, one can show, (\wg{and should probably prove by using
something like induction}) that the characteristic polynomial of $C_f$
is given by \[
	P_\alpha(\lambda) = 
 \lambda^n + \binom{\alpha}{1}\lambda^{n-1} + \binom{\alpha}{2}\lambda^{n-1} +
\cdots + \binom{\alpha}{n-1}\lambda + \binom{\alpha}{n} 
= (\lambda - \alpha_1) \cdots (\lambda - \alpha_n).
\]
Recalling that the eigenvalues of a matrix $A$ are given by the roots
of its characteristic polynomial, we manifestly have that if $f$ has
roots $\{\alpha_i\}$ then $C_f$ has eigenvalues $\{ \alpha_i \}$.

\end{solution}

\newpage

%------------------------- Problem 7 -----------------------

\begin{problem}
	\includegraphics[scale=0.8]{7.png}
	\hfill
\end{problem}

\begin{solution}
Recall that the conjugacy classes of $GL_4(\mathbb F_p)$ are those
matrices which are related to each other by conjugation. That is, the
classes of similar matrices over $GL_4(\F_p)$. Recall that every
similarity class of matrices has a unique rational canonical form. And
so, if we can enumerate the different possible rational canonical
forms we will have found the number of conjugacy classes
$GL_4(\F_p)$.

Recall that the rational canonical form of a matrix is of the form \[
	A = \begin{bmatrix}
		C_{f_1} & & & \\
		& C_{f_2} & & \\ 
		& & \cdots & \\ 
		& & & C_{f_k}
	\end{bmatrix},
\]
where the $C_{f_i}$ are the companion matrices of the invariant
	factors $f_i$. Recall that the invariant factors satisfy $f_1
	\vert f_2 \vert \cdots \vert f_k$. And so, we have cases given by
	increasing integer partitions of $4$, which will correspond to the
	possible degrees of the $f_i$: $1,1,1,1$, $1,1,2$, $2,2$,$1,3$,
	$4$. This follows a degree $n$ polynomial has an $n \times n$
	companion matrix and we have $A$ is $4 \times 4$. 

\textit{1,1,1,1:}
Let us start with the case where we have four invariant factors, all
of which are degree $1$. Since each $f_i$ is degree $1$ and monic, it
follows that we must in fact have $f_1 = f_2 = f_3 = f_4 =: f = x +
a_0$. In this case $A$ takes the form \[
	A = \begin{bmatrix}
		-a_0 & & & \\
		 & -a_0 & & \\
		 & & -a_0 &  \\
		 & & & -a_0
	\end{bmatrix}.
\]
In principle, we have one such distinct rational canonical form
	matrix for each element of $\F_p$.
	However, $A$ should be an element of $GL_4(\F_p)$ we should
	restrict to those $A$ which are invertible.
	In particular, we have $det(A) = (a_0)^4$ which is zero exactly
	when $a_0 = 0$ since $\F_p$ is in particular an integral domain.
	And so we have $p-1$ such rational canonical matrices of the above
	form.

\textit{1,1,2:}
We have three invariant factors $f_1 \vert f_2 \vert f_3$. However,
since $f_1 \vert f_2$ and both are monic, we must have $f_1 = f_2 = x
+ a_0$. Moreover, since $f_2 \vert f_3$ we have $f_3 = (x + a_0)(x+
b_0) = x^2 + (a_0 + b_0) + a_0b_0$. Then rational canonical form
matrix is given by \[
	A = \begin{bmatrix}
		-a_0 & & & \\
		 & -a_0 & & \\
		& & 0 & -a_0b_0 \\
		& & 1 & -(a_0 + b_0)
	\end{bmatrix}.
\]
	In principle we have $p$ choices for $a_0$ and $p$ choices for
	$b_0$, but we should restrict our choices to those which give
	invertible $A$.
	We have $det A = (-a_0)(-a_0)(a_0 b_0) = - (a_0)^3 b_0$. And so
	we should exclude those choices with $a_0 = 0$ or $b_0 = 0$.
This gives $(p-1)(p-1)$ rational canonical matrices of this form.

\textit{2,2:} We consider two invariant factors $f_1, f_2$ each of
degree two. Since $f_1 \vert f_2$ and both are monic and of equal
degree, we have $f_1 = f_2 := f = x^2 + a_1 x + a_0$. 
We then have \[
	A = 
	\begin{bmatrix}
		0 & -a_0 & 0 & 0 \\
		1 & -a_1 & 0 & 0 \\
		0 & 0 & 0  & -a_0 \\
		0 & 0  & 1 & -a_1 
	\end{bmatrix}.
\]
Now we exclude those $A$ which are non-invertible. 
	We have $det(A) = (a_0)^2$, and so we need to exclude the choices
	with $a_0 = 0$. 
Thus, we have $p(p-1)$ rational canonical matrices of this form.

\textit{1,3:} We have two invariant factors $f_1, f_3$ degree $1$ and
degree $3$ respectively and with $f_1 \vert f_3$. 
If we write $f_1 = x + a_0$ then we must have $f_3 = (x+a_0)(x^2 +
b_1x + b_0) = x^3 + (b_1 + a_0)x^2 + (b_0 + b_1a_0)x + a_0b_0$.
Then our rational canonical form in this case is given by 
\[
	A = \begin{bmatrix}
		-a_0 & 0 & 0 & 0 \\
		 0 & 0 & 0 & -a_0b_0 \\
		 0 & 1 & 0 & -(b_0 + b_1a_0) \\
		 0 & 0 & 1 & -(b_1 + a_0) 	
	\end{bmatrix},
\]
this matrix has determinant $det(A) = (a_0)^2b_0$. And so, we need to
restrict our choices to $a_0 \neq 0$ and $b_0 \neq 0$. 
This gives \wg{at most} $p(p-1)^2$ possible such rational canonical forms.
\wg{we should double check how much degeneracy there is in these choices.}

\textit{4:}
Lastly, we consider a single invariant factor $f = x^4 + a_3x^3 +
b_2x^2 + b_1x + b_0$ which has companion matrix \[
	A = \begin{bmatrix}
		0 & 0 & 0 & -a_0 \\
		1 & 0 & 0 & -a_1 \\
		0 & 1 & 0 & -a_2 \\
		0 & 0 & 1 & -a_3
	\end{bmatrix}.
\]
	This matrix has determinant $det(A) = a_0$. And so we should
	restrict our choices to $a_0 \neq 0$. This leaves us with
	$(p-1)p^3$ different such matrices $A$. Each one of these choices
	manifestly gives different matrices, and so we get exactly
	$(p-1)p^3$ different rational canonical matrices of this form.

Overall, counting up all the cases gives us \wg{less than or equal to} 
\[
	p-1 + (p-1)^2 + p(p-1) + p(p-1)^2 + (p-1)p^3
\]
distinct conjugacy classes in $GL_4(\F_p)$.

\end{solution}

\newpage

%------------------------- Problem 8 -----------------------

\begin{problem}[4]
	\hfill
\end{problem}

\begin{solution}
\end{solution}

\newpage

\end{document}
