\documentclass[12pt,letterpaper,boxed]{hmcpset}
\usepackage[margin=1in,headheight=14pt]{geometry}
\usepackage{amsfonts, amsmath, amssymb, enumerate, fancyhdr, gensymb, lastpage, mathtools, parskip, graphicx}
\usepackage{xcolor, tikz-cd}
\newcommand{\wg}[1]{\textcolor{violet}{#1}}
\newcommand{\OO}{\mathcal O}
\newcommand{\Q}{\mathbb Q}
\newcommand{\R}{\mathbb R}
\newcommand{\C}{\mathcal C}
\newcommand{\Z}{\mathbb Z}
\newcommand{\abs}[1]{\left|#1\right|}
\newcommand{\im}{\text{im }}
\newcommand{\inv}{^{-1}}
\newcommand{\Set}{\mathbf{Set}}
\newcommand{\normal}{\unlhd} %% one can also use \trianglelelefteq
\newcommand{\anglee}[1]{\langle #1 \rangle}
\usepackage[shortlabels]{enumitem}

% Numbering macros
\pagestyle{fancy}
\lhead{Will Gilroy}
\chead{Algs Homework \#}
\rhead{03 November 2021}
\lfoot{}
\cfoot{}
\rfoot{Page\ \thepage\ of\ \pageref{LastPage}}

\linespread{1.5}

\newcommand\blankpage{
    \thispagestyle{empty}
    \addtocounter{page}{-1}
    \newpage}
\renewcommand\footrulewidth{0.4pt}

\begin{document}

\problemlist{Algorithms HW } 

%------------------------- Problem 1 -----------------------

\begin{problem}
	\includegraphics[scale=0.8]{1.png}
	\hfill
\end{problem}

\begin{solution}
To show that the given square commuts, we must show that $(gf)(x) = (kh)(x)$ for
all $x \in \{1,2,3\}$.
Consider the image of $1 \in \{1,2,3\}$:
\begin{align*}
	gf(1) = g(1) = 1 =  k(1) = kh(1).
\end{align*}
Now, consider the image of $2 \in \{1,2,3\}$:
\begin{align*}
	gf(2) = g(2) = 1 = k(1) = kh(2).
\end{align*}
Finally, here's the image of $3$:
\begin{align*}
	gf(3) = g(4) = 3 = k(2) = kh(3).
\end{align*}
Hence, the given square commutes by definition. 
Note that we did not need to check whether $\im g = \im k$ (and in
fact these images in $\{1,2,3\}$ are not equal.)
\end{solution}

\newpage

%------------------------- Problem 2 -----------------------

\begin{problem}
	\includegraphics[scale=1]{2-1.png}
	\includegraphics[scale=1]{2-2.png}
	\hfill
\end{problem}

\begin{solution}
Suppose we have $(w,x) \otimes (y,z) \in \R^2 \otimes_\R \R^2$.
Following This element clockwise around the diagram we have that 
$(h \circ g)((w,x) \otimes (y,z)) = h(wy + xz)$ and following this
element counter-clockwise around the diagram we have 
$(g \circ f \otimes f)((w,x) \otimes (y,z)) = g((w - x, w+x) \otimes
(y - z, y + z)) = (w-x)(y-z) + (w+x)(y+z)$. 
That is, any $\R$-linear map $h: \R \to \R$ must satisfy \[
	h(wy + xz) = (w-x)(y-z) + (w+x)(y+z)
\]
for all $w,x,y,z \in \R$ \wg{is this last statement true, since our
inputs are tensor products and so there's some relation between these
symbols, right?}

Since $h$ is an $\R$-linear map we have that \[
	h(wy + xz) = h(1)(wy + xz).
\]
Moreover, since $\R$ is a rank $1$ free module over $\R$, we have
	that any $\R$-linear map $\R \to \R$ is determined by where it
	sends the basis $\{1\}$. 
Given the expression above we have that any such map $h$ satisfies 
\begin{align*}
	h(1) &= \frac{(w-x)(y-z) + (w+x)(y+z)}{wy + xz} \\
	&= \frac{wy -wz - xy + xz + wy + wz + xy + xz}{wy + xz} \\
	&= \frac{2(wy + xz)}{wy + xz} \\
	&= 2. 
\end{align*}
That is, there is a single map $h: \R \to \R$ which makes the above
diagram commute --- namely the one which sends the basis $1 \mapsto
2$, i.e $h(x) = 2x$. 

\wg{I'm curious if there's any geometric significance to this thing that we've just shown}

\end{solution}

\newpage

%------------------------- Problem 3 -----------------------

\begin{problem}
	\includegraphics[scale=0.8]{3.png}
	\hfill
\end{problem}
\begin{solution}
We need to show that $h''gf = g'f'h$. Suppose $x \in X$ \wg{If $\C$
is not ``sets with extra structure'' can we still reason about
functions by considering their actions on elements in their domain?
}

Consider the right-handed commuting square. Let $f(x) \in Y$. Since
this second square commutes, we have $h''gf = g'h'f$. Moreover, since
the left-handed square commutes, we have $h'f = f'h$. Substituting
this relation into our first equation gives us \[
	h''gf = g'h'f = g'f'h,
\]
as desired.
\end{solution}

\newpage

%------------------------- Problem 4 -----------------------

\begin{problem}
	\includegraphics[scale=0.9]{4-1.png}
	\includegraphics[scale=0.9]{4-2.png}
	\hfill
\end{problem}

\begin{solution}
For this question, recall that the universal properties of monic maps
and of epic maps. Let $\C$ be a category and let $X,Y \in \C$, then a
morphism $f: X \to Y$ is called monic if for all $Z \in \C$ and all
$g,h: Z \to X$ we have $fg = fh$ implies $g = h$. 
Likewise, $f: X \to Y$ is called epic if for all $Z \in \C$ and all
$g,h: Y \to Z$ we have $gf = hf$ implies $g = h$. 
\begin{itemize}
\item Suppose $f,g$ are monic and now consider $gf$. Let $W \in \C$
and suppose $h,k: W \to X$ such that $gfh = gfk$. Now, since $g$ is a
monomorphism and since function composition in $\C$ is 
associative, we have $g(fh) = gfh = gfk = g(fk)$ implies $fh = fk$.
Now, since $f$ is a monomorphism we have $h = k$. 
In other words, we have shown that $gfh = gfk$ implies $h = k$ for all
morphisms $h,k: Z \to X$. That is, $g,f$ monic imply that $gf$ is
monic.

\item 
Now suppose $f,g$ are epic and let $Z \in \C$ with $h,k: Z \to W$ such
that $hgf = kgf$. Since $f$ is epic we have that $(hg)f = hgf = kgf =
(kg)f$ implies $hg = kg$. Moreover, $g$ epic implies that $h = k$.
That is, we have $hgf = kgf$ implies $h = k$ and so $gf$ is epic.

\item 
Suppose $gf: X \to Z$ is a monomorphism. Let $W \in \C$ and $h,k: W
\to X$ such that $fh = fk$. We have that $\im(fh) = \im(fk) \in Y$ and
so, since $g = g$ we have that $gfh = gfk$. Now, since $gf$ is monic
we have that $h = k$. That is $fh = fk$ implies $h = k$, i.e. $f$ is
monic by definition. It is not necessary that $g$ be monic.

\item 
Now suppose $gf$ is epic. Let $W \in \C$ with $h,k: Z \to W$ such that
$hg = kg$. We have that $hgf = kgf$ as maps $X \to W$. But now, since
$gf$ is epic, we have that $h = k$. Thus $g$ is epic by definition. It
was not necessary that $f$ be epic.

\end{itemize}
\end{solution}

\newpage

%------------------------- Problem 5 -----------------------

\begin{problem}
	\includegraphics[scale=0.8]{5.png}
	\hfill
\end{problem}

\begin{solution}
Recall the axioms of a category. Given the objects and morphisms of
$\Set_G$ we need to verify $(1):$ that we have a well-defined
composition rule, i.e., that the given composition gives us a $\Set_G$
morphism, $(2)$ given our composition rule, that the given identity
morphism satisfies $g 1_X = 1_X g = g$ for all $g \in
Hom_{\Set_G}(X,X)$ for all $X \in \Set_G$, and $(3)$ that the given
composition rule is associative.

Firstly, the by theory of group actions, sets with group actions and
$G$-equivariant functions are a well defined collection of objects and
morphisms between those objects.

\textit{$(1)$ Composition:} Let $X,Y,Z \in \Set_G$ and let $f \in
Hom(X,Y)$ and $g \in Hom(Y,Z)$. Note that we have a well defined
function composition $gf$ from the category $\Set$. Now we must verify
that $gf$ is also $G$-equivariant. Since $f,g$ are $G$-equivariant we
have, for $\sigma \in G$ and $x \in X$, \[
	gf(\sigma x) = g(\sigma f(x)) = \sigma gf(x).
\]
And so, $gf \in Hom_\Set(X,Z)$ is $G$-equivariant by definition and
$gf$ is indeed a morphism in $Hom_{\Set_G}(X, Z)$. 

\textit{(2) Identity:} Let $X \in \Set_G$ and let $1_X: X \to X$ be the
identity function on $X$ as a set. Since $1_X$ is the identity
function for $X$ as a set we already have $g 1_X = 1_X g = g$ for all
functions
$g \in Hom_\Set(X,X)$. And so $1_X$, if it is $G$-equivariant, already
satisfies $g 1_X = 1_X g = g$ for all $g \in Hom_{\Set_G}(X,X)$. 
Let $\sigma \in G$ and consider \[
	1_X(\sigma x) = \sigma x = \sigma f(x),
\]
by definition of the action of $1_X$ as a function. And so, indeed,
$1_X$ is $G$-equivariant and so is a morphism in $Hom_{\Set_X}(X,X)$,
thus every $X \in \Set_X$ has an identity morphism.

\textit{(3) Associativity of function composition:} Note that
composition of functions is associative since $\Set$ is a category. It
follows immediately that $G$-equivariant function composition is
associative, since the $G$-equivariant functions from $X \to Y$ are a
``sub-class'' of the class of functions $X \to Y$. 

Now we show that finite products exists in $\Set_G$. 
\end{solution}

\newpage

%------------------------- Problem 6 -----------------------

\begin{problem}[4]
	\hfill
\end{problem}

\begin{solution}
\end{solution}

\newpage

%------------------------- Problem 7 -----------------------

\begin{problem}[4]
	\hfill
\end{problem}

\begin{solution}
\end{solution}

\newpage


%------------------------- Problem 8 -----------------------

\begin{problem}[4]
	\hfill
\end{problem}

\begin{solution}
\end{solution}

\newpage


%------------------------- Problem 4 -----------------------

\begin{problem}[4]
	\hfill
\end{problem}

\begin{solution}
\end{solution}

\newpage



\end{document}
