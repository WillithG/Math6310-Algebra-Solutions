\documentclass[12pt,letterpaper,boxed]{hmcpset}
\usepackage[margin=1in,headheight=14pt]{geometry}
\usepackage{amsfonts, amsmath, amssymb, enumerate, fancyhdr, gensymb, lastpage, mathtools, parskip, graphicx}
\usepackage{xcolor, tikz-cd}
\newcommand{\wg}[1]{\textcolor{violet}{#1}}
\newcommand{\OO}{\mathcal O}
\newcommand{\Q}{\mathbb Q}
\newcommand{\R}{\mathbb R}
\newcommand{\C}{\mathcal C}
\newcommand{\Z}{\mathbb Z}
\newcommand{\abs}[1]{\left|#1\right|}
\newcommand{\im}{\text{im }}
\newcommand{\inv}{^{-1}}
\newcommand{\normal}{\unlhd} %% one can also use \trianglelelefteq
\newcommand{\anglee}[1]{\langle #1 \rangle}
\usepackage[shortlabels]{enumitem}

% Numbering macros
\pagestyle{fancy}
\lhead{Will Gilroy}
\chead{Algs Homework \#}
\rhead{03 November 2021}
\lfoot{}
\cfoot{}
\rfoot{Page\ \thepage\ of\ \pageref{LastPage}}

\linespread{1.5}

\newcommand\blankpage{
    \thispagestyle{empty}
    \addtocounter{page}{-1}
    \newpage}
\renewcommand\footrulewidth{0.4pt}

\begin{document}

\problemlist{Algorithms HW } 

%------------------------- Problem 1 -----------------------

\begin{problem}[1]
	\hfill
\end{problem}

\begin{solution}
\end{solution}

\newpage

%------------------------- Problem 2 -----------------------

\begin{problem}
	\includegraphics[scale=1]{2-1.png}
	\includegraphics[scale=1]{2-2.png}
	\hfill
\end{problem}

\begin{solution}
Suppose we have $(w,x) \otimes (y,z) \in \R^2 \otimes_\R \R^2$.
Following This element clockwise around the diagram we have that 
$(h \circ g)((w,x) \otimes (y,z)) = h(wy + xz)$ and following this
element counter-clockwise around the diagram we have 
$(g \circ f \otimes f)((w,x) \otimes (y,z)) = g((w - x, w+x) \otimes
(y - z, y + z)) = (w-x)(y-z) + (w+x)(y+z)$. 
That is, any $\R$-linear map $h: \R \to \R$ must satisfy \[
	h(wy + xz) = (w-x)(y-z) + (w+x)(y+z)
\]
for all $w,x,y,z \in \R$ \wg{is this last statement true, since our
inputs are tensor products and so there's some relation between these
symbols, right?}

Since $h$ is an $\R$-linear map we have that \[
	h(wy + xz) = h(1)(wy + xz).
\]
Moreover, since $\R$ is a rank $1$ free module over $\R$, we have
	that any $\R$-linear map $\R \to \R$ is determined by where it
	sends the basis $\{1\}$. 
Given the expression above we have that any such map $h$ satisfies 
\begin{align*}
	h(1) &= \frac{(w-x)(y-z) + (w+x)(y+z)}{wy + xz} \\
	&= \frac{wy -wz - xy + xz + wy + wz + xy + xz}{wy + xz} \\
	&= \frac{2(wy + xz)}{wy + xz} \\
	&= 2. 
\end{align*}
That is, there is a single map $h: \R \to \R$ which makes the above
diagram commute --- namely the one which sends the basis $1 \mapsto
2$, i.e $h(x) = 2x$. 

\wg{I'm curious if there's any geometric significant to this thing that we've just shown}


\end{solution}

\newpage

%------------------------- Problem 3 -----------------------

\begin{problem}[3]
	\hfill
\end{problem}
\begin{solution}
\end{solution}

\newpage

%------------------------- Problem 4 -----------------------

\begin{problem}[4]
	\hfill
\end{problem}

\begin{solution}
\end{solution}

\newpage

%------------------------- Problem 5 -----------------------

\begin{problem}[4]
	\hfill
\end{problem}

\begin{solution}
\end{solution}

\newpage

%------------------------- Problem 6 -----------------------

\begin{problem}[4]
	\hfill
\end{problem}

\begin{solution}
\end{solution}

\newpage

%------------------------- Problem 7 -----------------------

\begin{problem}[4]
	\hfill
\end{problem}

\begin{solution}
\end{solution}

\newpage


%------------------------- Problem 8 -----------------------

\begin{problem}[4]
	\hfill
\end{problem}

\begin{solution}
\end{solution}

\newpage


%------------------------- Problem 4 -----------------------

\begin{problem}[4]
	\hfill
\end{problem}

\begin{solution}
\end{solution}

\newpage



\end{document}
