\documentclass[12pt,letterpaper,boxed]{hmcpset}
\usepackage[margin=1in,headheight=14pt]{geometry}
\usepackage{amsfonts, amsmath, amssymb, enumerate, fancyhdr, gensymb, lastpage, mathtools, parskip, graphicx}
\usepackage{xcolor, tikz-cd}
\newcommand{\wg}[1]{\textcolor{violet}{#1}}
\newcommand{\OO}{\mathcal O}
\newcommand{\Q}{\mathbb Q}
\newcommand{\R}{\mathbb R}
\newcommand{\C}{\mathcal C}
\newcommand{\Z}{\mathbb Z}
\newcommand{\abs}[1]{\left|#1\right|}
\newcommand{\im}{\text{im }}
\newcommand{\inv}{^{-1}}
\newcommand{\Set}{\mathbf{Set}}
\newcommand{\normal}{\unlhd} %% one can also use \trianglelelefteq
\newcommand{\anglee}[1]{\langle #1 \rangle}
\usepackage[shortlabels]{enumitem}

% Numbering macros
\pagestyle{fancy}
\lhead{Will Gilroy}
\chead{Algs Homework \#}
\rhead{03 November 2021}
\lfoot{}
\cfoot{}
\rfoot{Page\ \thepage\ of\ \pageref{LastPage}}

\linespread{1.5}

\newcommand\blankpage{
    \thispagestyle{empty}
    \addtocounter{page}{-1}
    \newpage}
\renewcommand\footrulewidth{0.4pt}

\begin{document}

\problemlist{Algorithms HW } 

%------------------------- Problem 1 -----------------------

\begin{problem}
	\includegraphics[scale=0.8]{1.png}
	\hfill
\end{problem}

\begin{solution}
To show that the given square commuts, we must show that $(gf)(x) = (kh)(x)$ for
all $x \in \{1,2,3\}$.
Consider the image of $1 \in \{1,2,3\}$:
\begin{align*}
	gf(1) = g(1) = 1 =  k(1) = kh(1).
\end{align*}
Now, consider the image of $2 \in \{1,2,3\}$:
\begin{align*}
	gf(2) = g(2) = 1 = k(1) = kh(2).
\end{align*}
Finally, here's the image of $3$:
\begin{align*}
	gf(3) = g(4) = 3 = k(2) = kh(3).
\end{align*}
Hence, the given square commutes by definition. 
Note that we did not need to check whether $\im g = \im k$ (and in
fact these images in $\{1,2,3\}$ are not equal.)
\end{solution}

\newpage

%------------------------- Problem 2 -----------------------

\begin{problem}
	\includegraphics[scale=1]{2-1.png}
	\includegraphics[scale=1]{2-2.png}
	\hfill
\end{problem}

\begin{solution}
Suppose we have $(w,x) \otimes (y,z) \in \R^2 \otimes_\R \R^2$.
Following This element clockwise around the diagram we have that 
$(h \circ g)((w,x) \otimes (y,z)) = h(wy + xz)$ and following this
element counter-clockwise around the diagram we have 
$(g \circ f \otimes f)((w,x) \otimes (y,z)) = g((w - x, w+x) \otimes
(y - z, y + z)) = (w-x)(y-z) + (w+x)(y+z)$. 
That is, any $\R$-linear map $h: \R \to \R$ must satisfy \[
	h(wy + xz) = (w-x)(y-z) + (w+x)(y+z)
\]
for all $w,x,y,z \in \R$ \wg{is this last statement true, since our
inputs are tensor products and so there's some relation between these
symbols, right?}

Since $h$ is an $\R$-linear map we have that \[
	h(wy + xz) = h(1)(wy + xz).
\]
Moreover, since $\R$ is a rank $1$ free module over $\R$, we have
	that any $\R$-linear map $\R \to \R$ is determined by where it
	sends the basis $\{1\}$. 
Given the expression above we have that any such map $h$ satisfies 
\begin{align*}
	h(1) &= \frac{(w-x)(y-z) + (w+x)(y+z)}{wy + xz} \\
	&= \frac{wy -wz - xy + xz + wy + wz + xy + xz}{wy + xz} \\
	&= \frac{2(wy + xz)}{wy + xz} \\
	&= 2. 
\end{align*}
That is, there is a single map $h: \R \to \R$ which makes the above
diagram commute --- namely the one which sends the basis $1 \mapsto
2$, i.e $h(x) = 2x$. 

\wg{I'm curious if there's any geometric significance to this thing that we've just shown}

\end{solution}

\newpage

%------------------------- Problem 3 -----------------------

\begin{problem}
	\includegraphics[scale=0.8]{3.png}
	\hfill
\end{problem}
\begin{solution}
We need to show that $h''gf = g'f'h$. Suppose $x \in X$ \wg{If $\C$
is not ``sets with extra structure'' can we still reason about
functions by considering their actions on elements in their domain?
}

Consider the right-handed commuting square. Let $f(x) \in Y$. Since
this second square commutes, we have $h''gf = g'h'f$. Moreover, since
the left-handed square commutes, we have $h'f = f'h$. Substituting
this relation into our first equation gives us \[
	h''gf = g'h'f = g'f'h,
\]
as desired.
\end{solution}

\newpage

%------------------------- Problem 4 -----------------------

\begin{problem}
	\includegraphics[scale=0.9]{4-1.png}
	\includegraphics[scale=0.9]{4-2.png}
	\hfill
\end{problem}

\begin{solution}
For this question, recall that the universal properties of monic maps
and of epic maps. Let $\C$ be a category and let $X,Y \in \C$, then a
morphism $f: X \to Y$ is called monic if for all $Z \in \C$ and all
$g,h: Z \to X$ we have $fg = fh$ implies $g = h$. 
Likewise, $f: X \to Y$ is called epic if for all $Z \in \C$ and all
$g,h: Y \to Z$ we have $gf = hf$ implies $g = h$. 
\begin{itemize}
\item Suppose $f,g$ are monic and now consider $gf$. Let $W \in \C$
and suppose $h,k: W \to X$ such that $gfh = gfk$. Now, since $g$ is a
monomorphism and since function composition in $\C$ is 
associative, we have $g(fh) = gfh = gfk = g(fk)$ implies $fh = fk$.
Now, since $f$ is a monomorphism we have $h = k$. 
In other words, we have shown that $gfh = gfk$ implies $h = k$ for all
morphisms $h,k: Z \to X$. That is, $g,f$ monic imply that $gf$ is
monic.

\item 
Now suppose $f,g$ are epic and let $Z \in \C$ with $h,k: Z \to W$ such
that $hgf = kgf$. Since $f$ is epic we have that $(hg)f = hgf = kgf =
(kg)f$ implies $hg = kg$. Moreover, $g$ epic implies that $h = k$.
That is, we have $hgf = kgf$ implies $h = k$ and so $gf$ is epic.

\item 
Suppose $gf: X \to Z$ is a monomorphism. Let $W \in \C$ and $h,k: W
\to X$ such that $fh = fk$. We have that $\im(fh) = \im(fk) \in Y$ and
so, since $g = g$ we have that $gfh = gfk$. Now, since $gf$ is monic
we have that $h = k$. That is $fh = fk$ implies $h = k$, i.e. $f$ is
monic by definition. It is not necessary that $g$ be monic.

\item 
Now suppose $gf$ is epic. Let $W \in \C$ with $h,k: Z \to W$ such that
$hg = kg$. We have that $hgf = kgf$ as maps $X \to W$. But now, since
$gf$ is epic, we have that $h = k$. Thus $g$ is epic by definition. It
was not necessary that $f$ be epic.

\end{itemize}
\end{solution}

\newpage

%------------------------- Problem 5 -----------------------

\begin{problem}
	\includegraphics[scale=0.8]{5.png}
	\hfill
\end{problem}

\begin{solution}
Recall the axioms of a category. Given the objects and morphisms of
$\Set_G$ we need to verify $(1):$ that we have a well-defined
composition rule, i.e., that the given composition gives us a $\Set_G$-morphism, $(2)$ given our composition rule, that the given identity
morphism satisfies $g 1_X = 1_X g = g$ for all $g \in
Hom_{\Set_G}(X,X)$ for all $X \in \Set_G$, and $(3)$ that the given
composition rule is associative.

Firstly, the by theory of group actions, sets with group actions and
$G$-equivariant functions are a well defined collection of objects and
morphisms between those objects.

\textit{$(1)$ Composition:} Let $X,Y,Z \in \Set_G$ and let $f \in
Hom_{\Set_G}(X,Y)$ and $g \in Hom_{\Set_G}(Y,Z)$. Note that we have a well defined
function composition $gf$ from the category $\Set$. Now we must verify
that $gf$ is also $G$-equivariant. Since $f,g$ are $G$-equivariant we
have, for $\sigma \in G$ and $x \in X$, \[
	gf(\sigma x) = g(\sigma f(x)) = \sigma gf(x).
\]
And so, $gf \in Hom_\Set(X,Z)$ is $G$-equivariant by definition and
$gf$ is indeed a morphism in $Hom_{\Set_G}(X, Z)$. 

\textit{(2) Identity:} Let $X \in \Set_G$ and let $1_X: X \to X$ be the
identity function on $X$ as a set. Since $1_X$ is the identity
function for $X$ we already have $g 1_X = 1_X g = g$ for all
functions
$g \in Hom_\Set(X,X)$. And so $1_X$, if it is $G$-equivariant,
satisfies the axiom for identity $G$-Set morphism. 
Let $\sigma \in G$ and consider \[
	1_X(\sigma x) = \sigma x = \sigma f(x),
\]
by definition of the action of $1_X$ as a function. And so, indeed,
$1_X$ is $G$-equivariant and so is a morphism in $Hom_{\Set_X}(X,X)$,
thus every $X \in \Set_X$ has an identity morphism.

\textit{(3) Associativity of function composition:} Note that
composition of functions is associative since $\Set$ is a category. It
follows immediately that $G$-equivariant function composition is
associative, since the $G$-equivariant functions from $X \to Y$ are a
``sub-class'' of the class of functions $X \to Y$. 

Now we show that finite products exists in $\Set_G$. 
We claim that binary products exist in $\Set_G$ (and so it will follow
that finite products exist in $\Set_G$ by iterating the construction
for binary products).
Recall that the universal property for binary products is the pullback
of the diagram $\cdot \leftarrow \cdot \rightarrow \cdot$. 
I claim that, given $X, Y \in \Set_G$, the cartesian product $X \times
Y$ with the usual projections $\pi_X, \pi_Y$ satisfy the universal property for $\Set_G$. To verify this
claim we have two things to show: $(1)$ that the cartesian product
has some $G$-action for which $\pi_X$ and $\pi_Y$ are $G$-equivariant
(i.e. that $X \times Y$ is indeed an object in $\Set_G$ and
$\pi_X: X \times Y \to X,\pi_Y: X \times Y \to Y$ are indeed morphisms
in $\Set_G$). And $(2)$, that $(X \times Y, \pi_X, \pi_Y)$ satisfy the
universal property of products

\textit{(1):} We show that $X \times Y$ has a $G$-action which makes
$\pi_X, \pi_Y$ into $G$-equivariant functions.
Recall that given $X, Y \in \Set_G$ we can construct the set 
$X \times Y := \{(x,y) : x \in X \quad y \in Y\}$. Define a
$G$-action on $X \times Y$ by $\sigma(x,y) = (\sigma x, \sigma y)$ for
$\sigma \in G$ and where $\sigma x, \sigma y$ are given by the $G$-action
structure on $X, Y$. 
We verify that this is indeed a $G$-action on $X \times Y$. 
Note that if $1 \in G$ is the identity element of $G$ then we have $1(x,y) = (1x, 1y)
= (x,y)$ for all $(x,y) \in X \times Y$. Moreover, if $\sigma, \delta
\in G$ we have $\sigma(\delta (x,y)) = \sigma(\delta x, \delta y)
= (\sigma\delta x, \sigma\delta y) = (\sigma\delta)(x,y)$. 
That is, our proposed action is indeed a $G$-action on $X \times Y$
and so the set $X \times Y$ is also an object in $\Set_G$. 

Now we verify that the projections $\pi_X: X \times Y \to X$ and
$\pi_Y: X \times Y \to Y$ are $G$-equivariant.
Let $(x,y) \in X \times Y$ and $\sigma \in G$ and consider \[
	\pi_X(\sigma(x,y)) = \pi_X((\sigma x, \sigma y)) = \sigma x = \sigma
\pi_X(x,y). 
\]
That is, indeed, $\pi_X$ is $G$-equivariant. Extremely similar
reasoning shows that $\pi_Y$ is $G$-equivariant. Hence $\pi_X, \pi_Y$
are indeed morphisms in $\Set_G$.

\text{$(2)$:} Finally, we need to show that $(X \times Y, \pi_X,
\pi_Y)$ satisfy the universal property of binary products. 
Recall that $(X \times Y, \pi_X, \pi_Y)$ satisfies the universal
property of binary products in $\Set$. That is, for any set $Z$ with
functions $f_X:Z\to X$ and $f_Y : Z \to Y$ we have a unique function
$f: Z \to X \times Y$ such that the following diagram commutes 
\[
\begin{tikzcd}
  & Z \arrow[ld, "f_X"'] \arrow[rd, "f_Y"] \arrow[d, "f", dashed] &   \\
X & X\times Y \arrow[l, "\pi_X"] \arrow[r, "\pi_Y"']              & Y
\end{tikzcd}
\]
Moreover, we know that $f$ necessarily has the form $f(z) =
(f_X(z),f_Y(z))$ for all $z \in Z$. 
If we verify that $f$ is $G$-equivariant, i.e. that it's a valid
morphism $Z \to X \times Y$ in $\Set_G$, then it will follow that $(X
\times Y, \pi_X, \pi_Y)$ satisfies the universal property in $\Set_G$
So, suppose now that $Z \in \Set_G$ is a set with a $G$ action, and
that $f_X: Z \to X$ and $f_Y: Z \to Y$ are $G$-equivariant functions. 
Indeed, let $\sigma \in G$, and consider \[
	f(\sigma z) = (f_X(\sigma z), f_Y(\sigma z)) 
	= (\sigma f_X(z), \sigma f_Y(z))
	= \sigma (f_X(z), f_Y(z)) 
	= \sigma f(z),
\]
by definition of $f$, by $G$-equivariance of $f_X, f_Y$, and by
definition of the $G$-action on $X \times Y$. 
That is, $f : Z \to X \times Y$ is a valid morphism in $\Set_G$ and it
follows that $f$ is the unique morphism which makes the following diagram
commute, now as a diagram in $\Set_G$
\[
\begin{tikzcd}
  & Z \arrow[ld, "f_X"'] \arrow[rd, "f_Y"] \arrow[d, "f", dashed] &   \\
X & X\times Y \arrow[l, "\pi_X"] \arrow[r, "\pi_Y"']              & Y
\end{tikzcd}
\]
Hence, $(X \times Y, \pi_X, \pi_Y)$ satisfies the universal property
of binary product in $\Set_G$. It follows then that finite products
exist in $\Set_G$.

\wg{Do empty products count as a finite product? I think very 
similar reasoning with products over a set would do the trick. Or,
if we really want, we can show that $\Set_G$ has an initial object,
and we know that we have unary products. }

\end{solution}

\newpage

%------------------------- Problem 6 -----------------------

\begin{problem}
	\includegraphics[scale=0.8]{6.png}
	\hfill
\end{problem}

\begin{solution}
\wg{still need to proofread this one}

We note for reference the definition of a functor here. A functor
$\mathcal F$ from
categories $C \to D$ is an association of objects from $C$ to objects
in $D$ such that $Hom_C(X,Y)$ has a corresponding association
$Hom_D(\mathcal F X, \mathcal F Y)$ for all $X, Y \in C$. 
Furthermore, this association of morphisms map identity morphisms to
identity morphisms, $\mathcal F 1_X = 1_{\mathcal F X}$ for all $X \in
C$, and should respect composition, $\mathcal F(g \circ f) = \mathcal
F g \circ \mathcal F f$ for all $f,g \in Hom_C(X,Y)$ for all $X,Y \in
C$. 

\begin{itemize}
\item 
We are given an association of $\textbf{Ring} \to \textbf{Set}$ by
associating a ring to its underlying set. We specify that $F$ acts on
morphisms by taking a ring homomorphism $R \to S$ to its underlying
function on the sets $R \to S$. 

Note that the identity morphism on a given ring then is associated to
the identity function on the underlying set, and so $F$ takes identity
morphisms to identity morphism. Moreover, recall that composition of
ring homomorphisms was defined as the composition of the underlying
functions (and then we verified that this was still a ring
homomorphism), but then it follows by definition that $F(g\circ f) \in
Hom_{Set}(FR,FT)$ is mapped to $Fg \circ Ff$, for all $F,T \in
\textbf{Ring}$. 

\item 
We are $G$ which takes a ring $R$ to its set of units $R^\times$, with a group
structure given by $R$-multiplication. Note that this is a
well-defined functor, since if $g \in R^\times$ then it has a
two-sided inverse by definition, and so $R^\times$ is closed under
inverses. Moreover, $R$ must contain a multiplicative identity $1$
which is also in $R^\times$ (it is its own two-sided inverse).
The given group operation is associative since ring multiplication is
associative.

If $f: R \to S$ is a ring homomorphism we associate $Gf: GR \to GS$ to
be the restriction to $R^\times$. We verify that this gives a well
defined group homomorphism $GR \to GS$.
First note that if $g \in R^\times$ with two-sided inverse $h \in
R^\times$ then we have $1 = f(1) = f(gh) = f(g)f(h) = f(h)f(g)$. That
is the image $f(g)$ has two-sided inverse $f(h)$ and so $f(R^\times)
\subseteq S^\times$. 
Moreover, informally, $f$ respects the group operation on $GR$ because $f$, by definition,
respects the ring multiplication on $R$. 
And so the association $(f:R \to S) \mapsto (Gf: GR \to GS)$ is a
valid association of morphisms in $\textbf{Ring}$ to morphisms in
$\textbf{Gp}$. 

Now we verify that that $G$ sends identity morphisms to identity
morphisms and that it respects composition. 
If $1_R$ is the identity morphism on a ring $R$, then $G 1_R$ is the
identity morphism on $GR$ since it is just the restriction onto a
subset of $R$. 
Moreover, if $g \circ f: R \to T$ is a composition then the
association  $G(g\circ f)$ is the restriction of $g \circ f$ to
$R^\times$. Moreover, unpacking the definitions as functions on sets
will show that 
$(g \circ f)\vert_{R^\times} = g\vert_{R^\times} \circ
f\vert_{R^\times}$. And so $G$ respects composition of morphisms.
Thus, $G$ is a functor. 


\item 
We have a functor from $\mathbf{Ring} \to \mathbf{Ring}$ by mapping $R
\to R^2$. We propose the following association of ring homomorphisms.
If we have a ring homomorphism $f: R \to S$ then associate 
$H f: R^2 \to S^2$ by $Hf(r_1, r_2) := (fr_1, f_2)$.
This definition is certainly a well defined map from $R^2$ to $S^2$.

Now, we verify that such an association satisfies the axioms of
functor. Suppose we have the identity morphism on a ring $1_R$ and
consider, for $r_1, r_2 \in R$, \[
	H1_R(r_1, r_2) = (1_R r_1, 1_R r_2) = (r_1, r_2).
\]
And so indeed, $H1_R$ is the identity morphism on $R^2$.
Now suppose we have a composition $g \circ f: R \to T$. Consider the
following \[
	H(g \circ f)(r_1, r_2) = ((g \circ f)r_1, (g \circ f)r_2)
		= Hg(fr_1, fr_2) 
		= (Hg)(Hf)(r_1, r_2). 
\]
That is, $H$ also respects ring homomorphism composition. Thus, the
given $H$ is indeed a functor. 

\item 
We are given a functor $\mathbf{Ring} \to \mathbf{Set}$. We propose
the following association of ring morphisms to set functions.

Suppose $f: R \to S$ is a ring homomorphism and then notice, if $x, y
\in R$ satisfy $y^2 = x^3 - x$ then we have \[
	f(y)^2 - f(x)^3 - f(x) = f(y^2 - x^3 - x) = f(0) = 0,
\]
That is $u := f(x), v:= f(y) \in S$ then satisfy $v^2 = u^3 - u$. 
This discussion then shows that the association 
$Kf: KR \to KS$ defined by $Kf(x,y) = (fx, fy)$ is a well defined set
function $KR \to KS$. 

We verify that this definition satisifies the axioms of functor.
Essentially the same discussion showing that $F,H$ are functors
applies to show that $K$ sends identity ring morphisms to the
identity function $KR \to KR$. Moreover, informally, the same computation showing
that $H$ respects composition also shows that $K$ respects
composition. 
And so $K$ is indeed a functor from $\textbf{Ring} \to \textbf{Set}$. 

\end{itemize}



\end{solution}

\newpage

%------------------------- Problem 7 -----------------------

\begin{problem}
	\includegraphics[scale=0.7]{7.png}
	\hfill
\end{problem}

\begin{solution}
For reference, a functor $K: \mathcal C \to \Set$ is
\textit{representable} if there exists an object $M \in \mathcal
C$ such  
that the functor $Hom_{\mathcal C}(M, -)$ is naturally isomorphic to
$K$. 

\begin{itemize}
\item 
We show that $Hom_{\textbf{Ring}}(\Z[x], -)$ is naturally isomorphic
to $F$.
First we note that maps $f: \Z[x] \to R$ are determined exactly by the
image of $x$ (since $1 \in \Z[x]$ must map to $1 \in R$ and the rest
of the images are given by using the ring homomorphism properties of
$f$).

Next we construct a natural transformation $Hom(\Z[x], -) \to F$.
The data of a natural transformation between these two functors is,
an association from each $R \in \textbf{Ring}$ to a morphism in $\Set$
$f_R: Hom(\Z[x], R) \to FR$. 
The above discussion gives an ``obvious'' function $Hom(\Z[x], R)
\xrightarrow \sim R$ in $\Set$, namely evaluating the map at $x$, $f
\mapsto f(x)$. 
Moreover, this function is in fact a bijection because
it has inverse 
\[
r \mapsto f(\ell)
:= \begin{cases}
	1 & \ell = 1 \\
	r & \ell = x \\
	\text{``extend using homomorphism property''} & \text{otherwise}
\end{cases}
\]

Let $g \in Hom_{\textbf{Ring}}(R, S)$. 
We are left to show that the above association $R \mapsto
f_R$ makes the following diagram commute in $\Set$
\[
\begin{tikzcd}
{Hom(\Z[x], R)} \arrow[r, "f_R"] \arrow[d, "g'"'] & R \arrow[d, "g"] \\
{Hom(\Z[x], S)} \arrow[r, "f_S"]                  & S               
\end{tikzcd}
\]
where $g'$ is given by the action of the functor $Hom(\Z[x], -)$ on $g:
R \to S$. 
Recalling that $f \in Hom(\Z[x], R)$ outputs to elements in $R$ we can
see that $g'(f) = g \circ f$ is a map $\Z[x] \to S$. 

We verify that this diagram commutes.
Let $f \in Hom(\Z[x], R)$. Following the top route gives the element \[
	f \mapsto f(x) \mapsto g(f(x)) = (g \circ f)(x). 	
\]
On the other hand, following the bottom route gives \[
	f \mapsto g \circ f \mapsto (g \circ f)(x).
\]
And so this diagram commutes.

Thus we have found a natural transformation $Hom(\Z[x], -) \to F$ with
each morphism $f_R$ a bijection. Thus we in fact have $F$ 
is naturally isomorphic to $\Hom(\Z[x], -)$ and so $F$ is a
representable functor, represented by $\Z[x]$.

\end{itemize}


\end{solution}

\newpage


%------------------------- Problem 8 -----------------------

\begin{problem}[4]
	\hfill
\end{problem}

\begin{solution}
\end{solution}

\newpage


%------------------------- Problem 4 -----------------------

\begin{problem}[4]
	\hfill
\end{problem}

\begin{solution}
\end{solution}

\newpage



\end{document}
