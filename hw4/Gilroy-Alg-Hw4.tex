\documentclass[12pt,letterpaper,boxed]{hmcpset}
\usepackage[margin=1in,headheight=14pt]{geometry}
\usepackage{amsfonts, amsmath, amssymb, enumerate, fancyhdr, gensymb, lastpage, mathtools, parskip, graphicx}
\usepackage{xcolor, tikz-cd}
\newcommand{\wg}[1]{\textcolor{violet}{#1}}
\newcommand{\OO}{\mathcal O}
\newcommand{\Q}{\mathbb Q}
\newcommand{\R}{\mathbb R}
\newcommand{\C}{\mathbb C}
\newcommand{\Z}{\mathbb Z}
\newcommand{\abs}[1]{\left|#1\right|}
\newcommand{\im}{\text{im }}
\newcommand{\inv}{^{-1}}
\newcommand{\normal}{\unlhd} %% one can also use \trianglelelefteq
\newcommand{\anglee}[1]{\langle #1 \rangle}
\newcommand{\Zfiveseven}{(\Z/5\Z) \otimes_\Z (\Z/7\Z)}
\newcommand{\Zsixeight}{(\Z/6\Z) \otimes_\Z (\Z/8\Z)}
\usepackage[shortlabels]{enumitem}

% Numbering macros
\pagestyle{fancy}
\lhead{Will Gilroy}
\chead{Algebra Homework \#4}
\rhead{01 Novemeber 2025}
\lfoot{}
\cfoot{}
\rfoot{Page\ \thepage\ of\ \pageref{LastPage}}

\linespread{1.5}

\newcommand\blankpage{
    \thispagestyle{empty}
    \addtocounter{page}{-1}
    \newpage}
\renewcommand\footrulewidth{0.4pt}

\begin{document}

\problemlist{Algebra Homework \#4} 

%------------------------- Problem 1 -----------------------

\begin{problem}
	\includegraphics[scale=0.8]{1.png}
	\hfill
\end{problem}

\begin{solution}
\end{solution}

\newpage

%------------------------- Problem 2 -----------------------

\begin{problem}
	\includegraphics[scale=0.8]{2.png}
	\hfill
\end{problem}

\begin{solution}
First we show that $(a)$ implies $(b)$. First notice that if every
submodule of $M$ is finitely generated then in particular $M$ is
finitely generated. \wg{note: at first I read $(a)$ incorrectly as
``$M$ is finitely generated'' and so I wrote this part of the question
under that assumption.}
And so, let $\{e_i\}$ be a finite, minimal, generating set for $M$ with $\# \{e_i\} =
n$. And let $N_1 \leq N_2 \leq N_3 \leq \cdots$ be an ascending chain
of $M$ submodules.

Let us construct a new chain $A_1 \subseteq A_2 \subseteq A_3
\subseteq \cdots $ where $A_i$ is some minimal generating set of
$N_i$. This is well defined since, as submodules of a finitely
generated module, each $N_i$ is also finitely generated.
We claim that the chain $A_1 \subseteq A_2 \subseteq \cdots$
stabilizes at some $k$.
Notice that the finite set $\{e_i\}$ is maximal amoungst the set of
$A_i$. And so the chain of $A_i$'s must stabilize either at $\{e_i\}$
or at some subset $B \subseteq \{e_i\}$. Let $k$ be the index where
stabilization occurs for the $A_i$.

In the first case where $A_k = \{e_i\}$ it follows then that $N_k =
M$, since $A_k$ was defined to be a minimal generating set for $N_k$.
It then follows that $N_i = N_k = M$ for all $i > k$. 
In the second case our chain of $A_i$'s stabilize with $A_k = B$.
In this case we claim that our chain of $N_i$ stabilizes at $N_k$
also. Let $m \in N_i$ for some $i > k$. Since $A_i$ is a generating
set for $N_i$ we have that $m$ is some finite $R$-linear combination
over $A_i$. However, $A_k = A_i$ and so the same finite $R$-linear
combination is also contained in $N_k$. In other words $m \in N_k$.
Hence $N_i \leq N_k$ and so $N_i = N_k$ for all $i > k$.

Now we show that $(b)$ implies $(a)$. 
Assume that $M$ satisfies the ascending chain condition for modules
and let $L \leq M$ be a submodule. Suppose, for contradiction, that
$L$ is not finitely generated. That is, $L$ has a minimal\footnote{
	Am I generally just allowed to ask for a minimal generating set?
}
 generating
set $\{\ell_i\}$ with $\# \{\ell_i\} = \infty$.
Suppose first that $\{\ell_i \}$ is countable.

We construct an asending chain $N_i := \{\sum_{k=1}^{i} r_i \ell_i :
r_i \in R \},$ submodules generated by the first $i$ generators of $L$.
We claim that $N_1 \leq N_2 \leq \cdots$ is an ascending chain which
does not stabilize. Consider $N_i$ for some $i$. 
Note that since $\{\ell_i\}$ is minimal, then there's some $k > i$
such that we cannot write $\ell_k$ as a finite $R$-linear combination
of elements in $\{\ell_j\}_{j=1}^{i}$. That is, there exists some
$N_k$ which contains an element $\ell_k$ which cannot be written as a
finite linear combiantion of elements in $N_i$. That is, we have found
a $k > i$ such that $N_k \neq
N_k$. Then our chain $N_1 \leq N_2 \leq \cdots$ does not stabilize.
However, this contradicts the submodule ascending chain condition on
$M$. Thus, it must be that $L$ is finitely generated.


\end{solution}

\newpage

%------------------------- Problem 3 -----------------------

\begin{problem}
	\includegraphics[scale=0.7]{3.png}
	\hfill
\end{problem}
\begin{solution}
\begin{itemize}
\item To show that $M/IM$ has the structure of an $R/I$-module, we
must show that $M/IM$ has an underlying abelian group structure with
respect to addition, and that there is a well-defined $R/I$ scaler
multiplication on $M/IM$.

First, we show that $M/IM$ has a well-defined abelian group structure
with respect to addition. By definition $M$ has an abelian group
structure, and so $IM$ is a normal subgroup of $M$. It follows that
the quotient group $M/IM$ is well-defined and moreover is also
abelian. One way to see that the quotient group is abelian is to
recall that the quotient map $\pi: M \to M/IM$ is a group homomorphism
and so \[
\pi(x) + \pi(y) = \pi(x + y) = \pi(y + x) = \pi(y) + \pi(x),
\]
for all $x,y \in M$. 

Next, we propose an $R/I$ scaler multiplication on $M/IM$. If $r + I
\in R/I$ and $m + IM \in M/IM$. Then I claim that scaler multiplication given
by $(r+I) \cdot (m + IM) := rm + IM$ is well defined.
Let $r + I = r' + I$ be equivalent elements of $R/I$. Then we have
that there's some $a \in I$ such that $r = r' + a$. Now let $m + IM
\in M/IM$ and consider \[
rm + IM = (r' + a)m + IM = r'm + IM + am + IM = r'm + IM + 0 + IM =
r'm + IM,
\]
since $am \in IM$ by definition. That is, our proposed scaler
multiplication maps $m \in M$ to the same class mod $IM$, regardless
of choice of representative of $r + IM$. Hence our multiplication is
well defined and we have found an $R/I$-module structure on $M/IM$.

\item 
Suppose $\phi: R^n \to R^m$ is an $R$-linear isomorphism. Since $R^n$
is a free $R$-module, it follows that $\phi$ is defined exactly by its
action on the basis $\{e_i\}$ where $e_i = (0, \cdots, 1, \cdots 0)$
with the $1$ in the $i$th position.

Now notice that $\bar e_i := \pi(e_i)$ is an $R/I$-basis for
$(R/I)^n$. And so maps out of $(R/I)^n$ can be defined by their action
on $\bar e_i$. We define a map $\bar \phi: (R/I)^n \to (R/I)^m$ by
$\bar \phi(\bar e_i) = (\pi \circ \phi)(e_i)$. This map is $R/I$
linear by construction, we claim that it is also a bijection.

First we show surjectivity. 
Suppose $(\bar b_1, \bar b_2, \cdots, \bar b_m) \in (R/I)^m$. Since
$\pi: R \to R/I$ is a surjection, there is an element $(b_1, b_2,
\cdots, b_m) \in R^m$ such that $\pi(b_1, b_2, \cdots, b_m) = (\bar
b_1, \bar b_2, \cdots, \bar b_m)$. Moreover, $\phi$ is an isomorphism
and so there also exists an element $(a_1, a_2, \cdots, a_n) \in R^n$
such that $\phi(a_1, \cdots, a_n) = (b_1, \cdots, b_m)$. It then
follows that $\bar\phi(\bar a_1, \bar a_2, \cdots, \bar a_n) = (\bar
b_1, \bar b_2, \cdots, \bar b_m)$. Thus, $\bar \phi$ is surjective.

Now consider $0 \in (R/I)^m$. Tuples of the form $(b_1, \cdots, b_m)
\in R^m$ with $b_i \in I$ map to $0$ under $\pi$. Since $\phi$ is an
isomorphism from $R$ to $R$, only elements $(a_1, \cdots, a_n) \in
R^n$ with $a_i \in I$ satisfy $\phi(a_1, \cdots, a_n) = (b_1, \cdots,
b_m)$ with $b_i \in I$. Then, only elements of the form
$(\bar a_1, \bar a_2, \cdots, \bar a_n) \in (R/I)^n$ map to $0 \in
(R/I)^m$ under $\bar \phi$. However, since $a_i \in I$ we have 
$(\bar a_1, \cdots, \bar a_n) = 0 \in (R/I)^n$. And so $\bar \phi$ is
injective.

We have found a bijective $R/I$-linear map $\bar\phi: (R/I)^n \to
(R/I)^m$ and so we have found an induced $R/I$-linear isomorphism
$(R/I)^n \to (R/I)^m$. 

\item 
Suppose $n = m$, then $R^n = R^m$ and so $R^n \cong R^m$ as
$R$-modules.

Now suppose that $R^n \cong R^m$. Since $R$ is a non-zero commutative
ring we have (via the axiom of choice (or perhaps Zorn's lemma)) 
that there exists a maximal ideal $I \normal R$. 
Now by part $(b)$ we have an induced isomorphism on the $R/I$ modules
$(R/I)^n \cong (R/I)^m$. However, since $I$ is maximal, it follows
that $R/I$ is a field and so $(R/I)^n$ and $(R/I)^m$ are in fact
$R/I$-vector spaces. Vector spaces are characterized by their
dimension and so it follows that $(R/I)^n \cong (R/I)^m$ as $R/I$
vector spaces implies $n = m$. 

Hence the rank of a free module over a non-zero commutative ring is a
well defined notion. 

\end{itemize}
\end{solution}

\newpage

%------------------------- Problem 4 -----------------------

\begin{problem}
	\includegraphics[scale=0.8]{4.png}
	\hfill
\end{problem}

\begin{solution}
\begin{itemize}
\item Consider the map $M \times N \to N \otimes_R M$ given by $((m,n)
\mapsto n \otimes m)$. Notice that this map is $R$-bilinear on $M \times
N$ since $\otimes: M \times N \to M \otimes_R N$ is.
Then by the universal property of tensor products we have a unique
$R$-linear map $f: M \otimes_R N \to N \otimes_R M$ such that $f \circ
\otimes = (m,n) \mapsto n \otimes m$. 
The same argument on the $R$-bilinear map $N \times M \to N \otimes_R
M$ given by $(n,m) \mapsto (m \otimes n)$ gives a unique $R$-linear
map $\tilde f: N \otimes_R M \to M \otimes_R N$ such that\footnote{
The ``$\otimes$'' in the following phrase now refers to the
$R$-bilinear map $N \times M \to N \otimes_R M$, whereas earlier it
referred to the $R$-bilinear map $M \times N \to M \otimes_R N$.
} $\tilde f
\circ \times = (n,m) \mapsto m \otimes n$. 

Notice now that $f \circ \tilde f = id_{N \otimes_R M}$ and $\tilde f
\circ f = id_{M \otimes_R N}$. Indeed $(f \circ \tilde f)(m \otimes n)
= f(n \otimes m) = m \otimes n$. And likewise for the other direction.

That is, we have found a bijective $R$-linear map $M \otimes_R N \to N
\otimes_R M$ and so in fact $M \otimes_R N$ is isomorphic to $N
\otimes_R M$.


\item Consider the following map $M \times N \to M$ via $(r,m) \mapsto
rm$ given by the module structure on $M$. Notice that this map is
bilinear \wg{come back and write the computation out}
And so by the universal property of tensor products we have a unique
$R$-linear map $f: R \otimes_R M \to M$ such that $r \otimes n
\mapsto rm$. 

I claim that this map is bijective and so is an isomorphism of
$R$-modules. First notice that $f$ is surjective. Indeed, if $m \in M$
then $f(1 \otimes m) = 1\cdot m = m$, so long as $R$ is not the zero
ring. If $R$ is the zero ring, then $M$ must be the zero module, and
then our desired isomorphism trivially holds. 

Next we show that $f$ is injective. 
Suppose we have $r \cdot m' = m$ for some $m,m' \in M$ and
$r \in R$. Then notice 
\[
	r \otimes m' = 1 \cdot r \otimes m' = r
		\otimes r \cdot m' = 1 \otimes m,
\]
And so $rm' = m$ implies $r \otimes m' = 1 \otimes m$. This suffices
to show that $f$ is injective because if generally we have $r_1 m_1 =
r_2 m_2$ then by definition $r_1 m_1 = m'$ for some $m' \in M$ and
then we have $m' = r_2 m_2$. 

Overall we have a bijective $R$-linear map $R \otimes_R M \to M$, and
so $R \otimes_R M \cong M$.

\item 
First we acquire an $R$-linear map $M \otimes_R (N \oplus L) \to (M
\otimes_R N) \oplus (M \otimes_R L)$ by leveraging the universal
property of tensor products. And then we show that this map is in fact
an isomorphism.

First define a map $h: M \times (N \oplus L) \to (M \otimes_R N)
\oplus (M \otimes_R L)$ by $h(m, (n,l)) = (m \otimes n, m \otimes l)$. 
Note that in $R$-Mod finite coproducts are also products, and so the
domain of the function $h$ is the $R$-module $M \oplus N \oplus L$.
We claim that this map is $R$-bilinear. 
Consider
\begin{align*}
	h(r_1 m_1 + r_2 m_2, (n,l)) 
		&= ((r_1m_1 + r_2m_2) \otimes n , (r_1m_1 + r_2 m_2) \otimes l) \\
		&= (r_1(m_1 \otimes n) + r_2 (m_1 \otimes n), r_1(m \otimes l) + r_2(m_2 \otimes n)) \\
		&= r_1(m_1 \otimes n, m_1 \otimes l_1) + r_2(m_2 \otimes n, m_2 \otimes l).
\end{align*}
By the definition of the tensor product relations, and of the
$R$-module structure on the direct sum of two $R$-modules. In other
words, we have shown that $h$ is $R$-linear in the first argument. Now
consider the second argument
\begin{align*}
	h(m, r_1(n_1, l_1) + r_2(n_2, l_2)) 
		&= (m \otimes r_1 n_1 + r_2 n_2, m \otimes r_1 l_1 + r_2 l_2) \\ 
		&= (r_1 (m\otimes n_1) + r_2 (m \otimes n_2), r_1(m \otimes l_1) + r_2(m \otimes l_2)) \\
		&= r_1(m \otimes n_1, m \otimes l_1) + r_2(m \otimes n_2, m \otimes l_2), 
\end{align*}
using the bilineararity of the tensor product, and by using the
definition of the $R$-module structure on the direct sum of
$R$-modules. That is, we've shown $h$ is $R$-linear in the second
argument also. And so $h$ is an $R$-bilinear map.

We are now free to use the universal property of tensor products to
acquire a new $R$-linear map $\phi: M \otimes_R (N \oplus L) \to (M
\otimes_R N) \oplus (M \otimes_R L)$. where $\phi(m \otimes (n,l)) = (m \otimes
n, m \otimes l)$.

Next, we will construct an $R$-linear map $\psi: (M \otimes_R N)
\oplus (M \otimes_R L) \to M \otimes_R (N \otimes L)$ using the
universal property of coproducts. We have a commutative diagram
\[
\begin{tikzcd}
M \times N \arrow[d, "\otimes"'] \arrow[r, "f_1"]          & M \otimes_R (N \oplus L)                                         & M \times L \arrow[d, "\otimes"] \arrow[l, "f_2"']           \\
M \otimes_R N \arrow[r, hook] \arrow[ru, "\psi_1", dashed] & (M \otimes_R N) \oplus (M \otimes_R L) \arrow[u, "\psi", dashed] & M \otimes_R L \arrow[l, hook] \arrow[lu, "\psi_2"', dashed]
\end{tikzcd}
\]
where the maps $\psi_1, \psi_2$ will be built out of the universal
property of tensor products, and the desired map $\psi$ will
consequently be determined by the universal property of corpoduct in
$R$-mod. First, we must specify bilinear maps $f_1: M\times N \to
M\otimes_R (N \oplus L)$ and $f_2: M \times L \to M \otimes_R (N
\oplus L)$. 

I claim that the map $f_1(m, n) = m \otimes (n, 0)$ is bilinear.
If $n$ is fixed then for each $r_1, r_2 \in R$ and $m_1, m_2 \in M$ we
have 
$f_1(r_1m_1 + r_2m_2, n) = (r_1 m_2 + r_2 m_2)
\otimes (n, 0) = r_1(m_1 \otimes (n,0)) + r_2 (m_2 \otimes (n,0)) =
r_1f_1(m_1, n) + r_2f_2(m_2, n)$,
by the bilinearity of tensor product. I.e. $f_1$ is $R$-linear in its
first argument.
Moreover, if we fix $m \in M$ we have 
\begin{align*}
f_1(m, r_1 n_1 + r_2 n_2) &= m \otimes (r_1n_1 + r_2 n_2, 0) \\
	&= m \otimes [r_1(n_1, 0) + r_2(n_2, 0)] \\
	&= r_1(m \otimes (n_1, 0)) + r_2(m \otimes (n_2, 0)) 
	&= r_1f_1(m, n_1) + r_2f_1(m, n_2).
\end{align*}
And so, $f_1$ is $R$-linear in its second argument.
Hence, $f_1: M \times N \to M \otimes_R (N \oplus L)$ is $R$-bilinear.
A very similar calculation will show that $f_2: M \times L \to M
\otimes_R (N \oplus L)$ is also $R$-bilinear.

Then, by the universal property of tensor products, we have $R$-linear
maps $\psi_1: M \otimes_R N \to M \otimes_R (N \oplus L)$ and $\psi_2:M
\otimes \otimes_R L \to M \otimes_R (N \oplus L)$ with $\psi_1(m
\otimes n) = m \otimes (n, 0)$ and $\psi_2(m \otimes l) = m \otimes
(0, l)$. Then by the universal property of coproducts, we have an
$R$-linear map $\psi: (M \otimes_R N ) \oplus (M \otimes_R L ) \to M
\otimes_R (N \oplus L)$ where $\psi(m_1 \otimes n, m_2 \otimes l) = 
m_1 \otimes (n, 0) + m_2 \otimes (0, l)$. 
We claim that $\psi$ is an $\R$ linear map which is inverse to $\phi$. 
If we show that it will then follow that $\phi$ is in fact an
isomorphism $M \otimes_R (N \oplus L) \to (M \otimes_R N) \oplus (M
\otimes_R L)$. 

First let $m \otimes (n,l) \in M \otimes_R (N \oplus L)$ and consider
\begin{align*}
	(\psi \circ \phi)(m \otimes (n,l))
		&= \phi(m \otimes n, m \otimes l) \\
		&= m \otimes (n, 0) + m \otimes (0, l) \\
		&= m \otimes [(n,0) + (0, l)] \\
		&= m \otimes (n,l).
\end{align*}
Now let $(m_1 \otimes n, m_2 \otimes l) \in (M \otimes_R N) \oplus (M
\otimes_R L)$ and consider 
\begin{align*}
	(\phi \circ \psi)(m_1 \otimes n, m_2 \otimes l) 
		&= \phi(m_1 \otimes (n,0) + m_2 \otimes (0, l)) \\
		&= \phi(m_1 \otimes (n,0)) + \phi(m_2 \otimes (0,l)) \\
		&= (m_1 \otimes n, m_1 \otimes 0) + (m_2 \otimes 0, m_2 \otimes l) \\
		&= (m_1 \otimes n, 0) + (0, m_2 \otimes l) \\
		&= (m_1 \otimes n, m_2 \otimes l).
\end{align*}
And so, $\phi, \psi$ are inverse $R$-linear homomorphisms and 
$M \otimes_R (N \oplus L) \cong (M \otimes_R N) \oplus (M \otimes_R L)$.


\end{itemize}
\end{solution}

\newpage

%------------------------- Problem 5 -----------------------

\begin{problem}
	\includegraphics[scale=0.8]{5.png}
	\hfill
\end{problem}

\begin{solution}
\begin{enumerate}[(a)]
\item 
\item b

\item We show that $\Zfiveseven \cong 0$. First notice that $gcd(5,7)
= 1$ and so by Bezout's identity there exists integers $x, y$ such
that $5x + 7y = 1$. 

Moreover, every element of $\Zfiveseven$ can be written as a pure
tensor since our factors are cyclic, and so, let $[p] \otimes [q] \in
\Zfiveseven$. Recalling modular arithmetic rules and distributivity of the
tensor product we have
\begin{align*}
	[p] \otimes [q] &= [1] \otimes [pq] \\
	&= [5x + 7y] \otimes [pq] \\
	&= ([5x] + [7y]) \otimes [pq] \\
	&= [5x] \otimes[pq] + [7y] \otimes [pq] \\
	&= [5x] \otimes [pq] + [y] \otimes [7pq] \\
	&= 0 + 0 \\
	&= 0.
\end{align*}
Thus, we have shown that every elements in $\Zfiveseven$ is actually
the zero tensor. In other words $\Zfiveseven = 0$. 

\item We show that $\Zsixeight \cong \Z/2\Z$. First we show that every
element in $\Zsixeight$ can is either the $0$ element or $[1] \otimes
[1]$. Since $\Zsixeight$ has cyclic factors, we can write every
element of $\Zsixeight$ as a pure tensor. 

Let $[p] \otimes [q] \in
\Zsixeight$. There exist integers $x,y \in \Z$ such that $6x + 8y =
gcd(6,8) = 2$. We have two cases either $[p] = [2k]$ or $[p] = [2k+1]$
for $k = 0,1,2$. Consider the first case, we have 
\begin{align*}
	[2k] \otimes [q] &= 2([k] \otimes [q]) \\
		&= (6x + 8y)([k] \otimes [q]) \\
		&= ([6xk] \otimes [q]) + ([k] \otimes [8yq]) \\
		&= 0.
\end{align*}
Now in the second case, using similar reasoning, we have 
\begin{align*}
	[2k+1] \otimes [q] &= ([2k] \otimes [q]) + ([1] \otimes [q]) 
		= [1] \otimes [q].
\end{align*}
Now, we have two additional cases, either $[q] = [2\ell]$ or $[q] =
[2\ell + 1]$ for $\ell = 0,1,2,3$. Again, using similar reasoning, if
$[q] = [2k]$ then we have $[1] \otimes [q] = [1] \otimes [2k] = 0$.
And in the second case we have 
\begin{align*}
	[1] \otimes [2k + 1] &= [1] \otimes [2k] + [1] \otimes [1] \\
	&= 0 + [1] \otimes [1] \\
	&= [1] \otimes [1].
\end{align*}
Thus we have shown that every element in $\Zsixeight$ is either the
$0$ tensor or $[1] \otimes [1]$.

Now, we claim that $\Zsixeight = \Z/2\Z$. To show so this we construct
a bijective $R$-linear map $\Zsixeight = \Z/2\Z$. I claim that $\phi:
\Zsixeight \to \Z/2\Z$ via $\phi(0) = 0$ and $\phi([1] \otimes [1]) =
[1]$ is such an isomorphism. \wg{Okay, to show this for real, we need
to show three things, that this map is: $R$-linear, well-defined
with respect to the tensor product realtions, and that this map is
bijective. The other way to do this is using the universal property of
tensor products, but we cannot write elements $(p,q) \in \Zsixeight$ 
as $0$ or $(1,1)$, like we can for elements in the tensor product. So,
I'm not sure this is any easier in this case.}

\item \wg{note: pretty sure this is isomorphic to $\Z/2\Z$. write this
up}
\end{enumerate}
\end{solution}

\newpage

%------------------------- Problem 6 -----------------------

\begin{problem}
	\includegraphics[scale=0.7]{6.png}
	\hfill
\end{problem}

\begin{solution}
We define the desired map by using the universal property of tensor
products.
Consider the following composition
\[
	M \oplus N \xrightarrow{(f,g)} M' \oplus N'  \xrightarrow{\otimes} M' \otimes_R N',
\]
via 
\[
	(m,n) \mapsto (f(m), g(n)) \mapsto f(m) \otimes g(n).
\]
We claim that this composition $\phi: M \oplus N \to M \otimes_R N$ is
$R$-bilinear. Indeed, first let $n \in N$ be fixed and then consider
\begin{align*}
	\phi(r_1 m_1 + r_2 m_2, n)  
		&= (r_1 f(m_1) + r_2 f(m_2)) \otimes g(n) \\
		&= (r_1 f(m_1)) \otimes g(n) + (r_2 f(m_2)) \otimes g(n) \\
		&= r_1(f(m_1) \otimes g(n)) + r_2 (f(m_2) \otimes g(n)) \\
		&= r_1 \phi(m_1, n) + r_2 \phi(m_2, n), 
\end{align*}
by the relations on the elements of the tensor product. In other
words, we have shown that $\phi(-, n)$ is $R$-linear for each $n \in N$. A very similar
calculation will show that $\phi(m, -)$ is $R$-linear for each $m \in
M$. That is, $\phi$ is $R$-bilinear.

By the universal property of the tensor product we then have an
$R$-linear map $\psi: M \otimes_R N \to M' \otimes_R N$ given by $m
\otimes n \mapsto f(m) \otimes g(n)$, as desired. 

\end{solution}

\newpage

%------------------------- Problem 7 -----------------------

\begin{problem}
	\includegraphics[scale=0.8]{7.png}
	\hfill
\end{problem}

\begin{solution}
\begin{itemize}
\item I claim that $1 \otimes e_1, 1 \otimes, e_2, \cdots, 1 \otimes
e_n \in \C \otimes_\R V$ is a basis. First we show that these tensors
span $\C \otimes_\R V$. First let $\alpha \otimes v \in \C \otimes_\R
V$ be a pure tensor. Then, since $\{e_i\}$ is a basis for $V$ we have
$v = \lambda_1 e_1 + \cdots \lambda_n e_n$ for $\lambda_i \in \R$.
Now notice 
\begin{align*}
	\alpha\lambda_1(1 \otimes e_1) + \alpha\lambda_2(1 \otimes e_2)
		+ \cdots + \alpha\lambda_n(1 \otimes e_n) 
	&= \alpha 
	\left[ 1 \otimes \lambda_1 e_1 + 1 \otimes \lambda_2 e_2 + \cdots 1 \otimes \lambda_n e_n \right] \\
	&=  \alpha \otimes (\lambda_1 e_1 + \cdots + \lambda_n e_n) \\
	&=  \alpha \otimes v.
\end{align*}
That is, every pure tensor in $\C \otimes_R V$ is a finite $\C$-linear
combination of the $1 \otimes e_i$. It then follows that every element
of $\C \otimes_\R V$, which is a finite $\C$-linear combination of
pure tensors, is also a finite $\C$-linear combination of the $1
\otimes e_i$. Thus, the $1 \otimes e_i$ span $\C \otimes_R V$ over
$\C$. 

Later we will show that the $\{1 \otimes e_i\}$ are also
$\C$-linearly independent.

\item 
We want to study the $\C$-linear maps 
$\C \otimes_\R V \to \C = \C \otimes_\R \R$. We first consider maps
which descend from bilinear maps $C \times V \to  \C \otimes_\R \R$
induced by $f \in V^*$.
First, fix a map $f \in
V^\star$. And then we define a map $\phi_f: \C \times V \to  \C
\otimes_\R \R$ via $\phi_f(\alpha, v) = \alpha \otimes f(v)$.
First notice that this map is $\R$-linear in the second coordinate: if
we fix $\alpha \in \C$ and let $a_1, a_2 \in \R$ indeed, we have \[
\phi_f(\alpha, a_1v_1 + a_2 v_2) = \alpha \otimes f(a_1 v_1 + a_2 v_2)
	= \alpha \otimes (a_1 f(v_1) + a_2 f(v_2)) 
	= a_1(\alpha \otimes f(v_1)) + a_2(\alpha \otimes f(v_2)), 
\]
Since $f$ is an $\R$-linear map and by the bilinearity of the tensor
product.
A similar calculation shows that $\phi_f$ is $\C$-linear in the first
coordiante.

Now, since $\delta_1, \cdots, \delta_n$ is a basis for $V^*$ then we
can represent $f(v) = a_1 \delta_1(v) + a_2 \delta_2(v) + \cdots a_n
\delta_n(v)$ for $a_i \in \R$. Then we can represent
\[
\phi_f(\alpha, v) = \alpha \otimes (a_1\delta_1(v) + \cdots + a_n \delta_n(v))
	= \alpha a_1 (1 \otimes \delta_1(v)) + \alpha a_2 (1 \otimes \delta_2(v)) +
		\cdots + \alpha a_n (1 \otimes \delta_n(v)),
\]
Note that the $a_i$ in the computation above depends only on the map
$f \in V^*$. 
This suggests that a plausible basis for $Hom_\C(\C \otimes_\R V, \C)$
could be the maps $1 \otimes \delta_i \in Hom_\C(\C \otimes_\R, \C).$

\wg{Generally though, however, we need to show that an arbitrary
$\C$-linear function $\C \otimes_\R V \to \C$ can be written as a
finite linear combination of the $\{1 \otimes \delta_i\}$}.

\item See the previous parts to see some partial arguments that the
elements given are $R$-linear bases for their corresponding spaces.
I need to show more to show that they are in fact $\C$-linear bases
for their corresponding spaces.

I will, however, show that $\{1 \otimes e_i \} \subseteq \C \otimes_\R
V$ and $\{1 \otimes
\delta_i \} \subseteq Hom_\C(\C \otimes_\R V, \C)$ are dual to each other. Consider \[
(1 \otimes \delta_i)(1 \otimes e_i) = 1 \otimes \delta_i(e_i) = 1
\otimes 1 = 1 \in \C.
\]
Whereas, for $i \neq j$  \[
(1 \otimes \delta_i)(1 \otimes e_j) = 1 \otimes \delta_i(e_j) = 1
\otimes 0 = 0 \in \C. 
\]
This shows that $\{1 \otimes \delta_i\}$ and $\{1 \otimes e_i \}$ are
dual and in fact orthonormal.

Earlier we showed that $\{1 \otimes e_i \}$ spanned $\C \otimes_\R V$.
Now,
we show that $\{1 \otimes e_i\}$ are $\C$-linearly independent.
Suppose we have $\lambda_1(1 \otimes e_1) + \lambda_2(1 \otimes e_2) +
\cdots + \lambda_n(1 \otimes e_n) = 0$ for $\lambda_i \in \C$.
Applying the $\C$-linear homomorphism $1 \otimes \delta_1$ gives
\begin{align*}
	\lambda_1(1 \otimes \delta_1)(1 \otimes e_1) 
	+ \lambda_2(1 \otimes \delta_2)(1 \otimes e_2) 
	+ \cdots + \lambda_n(1 \otimes \delta_n)(1 \otimes e_n)
	&= (1 \otimes \delta_n)(0) \\
	\lambda_1(1) + 0 + \cdots + 0 &= 0 \\
	\lambda_1 &= 0.
\end{align*}
Likewise, applying $1 \otimes \delta_i$ for each $i$ will then give
$\lambda_i = 0$ for each $i$. Thus, $\{1 \otimes e_i\}$ is $\C$-linearly
independent by definition.

Similarly, earlier we showed that the functions $\{1 \otimes
\delta_i\}$ spanned the $\C$-module $Hom(\C \otimes_\R V, \C = \C
\otimes_\R \R)$. Now we show that they are $\C$-linearly independent.
Suppose $\lambda_1(1 \otimes \delta_i) + \lambda_2(1 \otimes \delta_2)
+ \cdots \lambda_n(1 \otimes \delta_n) = 0$ sum to the zero function. 
Recall that these functions act by $(1 \otimes \delta_i)(\alpha
\otimes v) = \alpha \otimes \delta_i(v) \mapsto \alpha\delta_i(v) \in
\C$. So now plugging in the element $1 \otimes e_1$ into both sides
gives 
\begin{align*}
	\lambda_1(1 \otimes \delta_i)(1 \otimes e_1) 
	+ \lambda_2(1 \otimes \delta_2)(1 \otimes e_2)
	+ \cdots \lambda_n(1 \otimes \delta_n)(1 \otimes e_n) &= 0(1 \otimes e_1) \\
	\lambda_1(1 \cdot 1) + \lambda_2(1 \cdot 0) + \cdots \lambda_n(1 \cdot 0) &= 0 \\
	\lambda_1 &= 0.
\end{align*}
Likewise, plugging in the elements $1 \otimes e_i$ will give that
$\lambda_i = 0 $ for each $i$. In other words, the $\{1 \otimes
\delta_i\}$ are $\C$ linearly independent.

\end{itemize}

\end{solution}

\newpage

%------------------------- Problem 8 -----------------------

\begin{problem}
	\includegraphics[scale=0.8]{8.png}
	\hfill
\end{problem}

\begin{solution}
\end{solution}

\newpage



\end{document}
