\documentclass[12pt,letterpaper,boxed]{hmcpset}
\usepackage[margin=1in,headheight=14pt]{geometry}
\usepackage{amsfonts, amsmath, amssymb, enumerate, fancyhdr, gensymb, lastpage, mathtools, parskip, graphicx}
\usepackage{xcolor, tikz-cd}
\newcommand{\wg}[1]{\textcolor{violet}{#1}}
\newcommand{\OO}{\mathcal O}
\newcommand{\Q}{\mathbb Q}
\newcommand{\R}{\mathbb R}
\newcommand{\C}{\mathcal C}
\newcommand{\Z}{\mathbb Z}
\newcommand{\abs}[1]{\left|#1\right|}
\newcommand{\im}{\text{im }}
\newcommand{\inv}{^{-1}}
\newcommand{\normal}{\unlhd} %% one can also use \trianglelelefteq
\usepackage[shortlabels]{enumitem}

% Numbering macros
\pagestyle{fancy}
\lhead{Will Gilroy}
\chead{Algebra Homework \#1}
\rhead{01 September 2025}
\lfoot{}
\cfoot{}
\rfoot{Page\ \thepage\ of\ \pageref{LastPage}}

\linespread{1.5}

\newcommand\blankpage{
    \thispagestyle{empty}
    \addtocounter{page}{-1}
    \newpage}
\renewcommand\footrulewidth{0.4pt}

\begin{document}

\problemlist{Math6310 Algebra Homework \#1 } 

%------------------------- Problem 1 -----------------------

\begin{problem}
	\includegraphics[scale=0.9]{1.png}
	\hfill
\end{problem}

\begin{solution}
\begin{enumerate}
	\item We show that $\gamma_g$ is a bijective homomorphism, for
	some fixed $g \in G$.
	Let $k, \ell \in G$ then we have \[
	\gamma_g(k \cdot \ell) = g \cdot
	(k \cdot \ell) \cdot g\inv = g \cdot k \cdot e \cdot \ell \cdot
	g\inv = (g \cdot k \cdot g\inv) \cdot (g \cdot \ell \cdot g\inv) =
	\gamma_g(k) \cdot \gamma_g(\ell),
	\]
	since group products are associative, and by definition of the
	identity element. Hence $\gamma_g$ is a homomorphism for all $g
	\in G$.

	Now suppose $\gamma_g(h) = e$ for some $h \in G$ we have 
	\begin{align*}
		\gamma_g(h) &= e \\
		ghg\inv &= e \\
		(g\inv g)h(g\inv g) &= g\inv e g\\
		h &= g\inv e g \\ 
		h &= e.
	\end{align*} 
	Thus, $\gamma_g(h)$ is injective. Now let $k \in G$ and notice
	that $\gamma_g(g\inv k g) = g\cdot g\inv k g\inv g = k$. Moreover,
	$g\inv k g \in G$ since $G$ is closed under its group operation.
	That is, $\gamma_g$ is surjective for all $g \in G$.
	Hence, we have shown that $\gamma_g$ is an automorphism of $G$.

	\item Let $g,h \in G$.
	And let $f: G \to Aut(G)$ be the map $f(g)
	= \gamma_g$. 
	
	Consider the action of $\gamma_{gh}$ on
	some group element $k$. We have
	\begin{align*}
		\gamma_{gh}(k) &= (gh) k (gh)\inv \\
			&= (gh) k (h\inv g\inv) \\ 
			&= g (h k h\inv) g\inv \\
			&= (\gamma_g \circ \gamma_h)(k),
	\end{align*}
	holds for all $k \in G$. That is, we have shown $f(g \cdot h) = f(g)
	\circ f(h)$, where $\cdot$ denotes the product in $G$ and $\circ$
	denotes function composition --- the group operation in $Aut(G)$.
	Hence, $f$ is a homomorphism.

	\item We show directly that $\im f$ is closed under conjugation by
	homomorphism in $Aut(G)$. Let $h \in Aut(G)$ and $\gamma_g \in \im
	f$. There then exists an
	inverse homomorphism $h\inv$, consider the action of \[
		h \circ \gamma_g \circ h\inv.
	\]
	This is an automorphism since the composition of group
	homomorphisms is again a group homomorphism.

	Let $k \in G$ and consider 
	\begin{align*}
		(h \circ \gamma_g \circ h\inv)(k) &= h(g \cdot h\inv(k) \cdot g\inv) \\
			&= h(g) \cdot k \cdot h(g\inv), && \text{since $h$ is a homomorphism}
	\end{align*}
	Moreover, $h(g) = g' \in G$ since $h$ is an automorphism of $G$.
	That is, we have shown $(h \circ \gamma_g \circ g\inv) = f(g') \in
	\im f$. And so, $\im f$ is a normal subgroup of $Aut(G)$ by
	definition.

\end{enumerate}
\end{solution}

\newpage

%------------------------- Problem 2 -----------------------

\begin{problem}
	\includegraphics[scale=0.8]{2.png}
	\hfill
\end{problem}

\begin{solution}
\wg{Big Apologies}
\end{solution}

\newpage

%------------------------- Problem 3 -----------------------

\begin{problem}
	\includegraphics[scale=0.8]{3.png}
	\hfill
\end{problem}
\begin{solution}
We search for the conjugacy classes of $S_n$ whose elements are even
permutations. Note that this is a well-defined notion, since if
$\sigma, \tau \in S_n$ are even permutations then $\sigma \tau
\sigma\inv$ has a transposition decomposition whose length is a
product of three even numbers, and is thus even.

Recall that the conjugacy classes are exactly given by the type of a
permutation and that the number of valid conjugacy classes
correspond to the number of partitions of $n$. And so, we will
enumerate the partitions of $6$ and then acquire the conjugacy classes
of $A_6$ by choosing the classes which correspond to even partitions.
The partitions of $6$ are:
\begin{align*}
	\mathbf{[1,1,1,1,1,1,1]} &\qquad [2,2,2] \qquad \mathbf{[2,2,1,1]} \qquad [2,1,1,1,1]
	\qquad \mathbf{[3,3]} \qquad [3,2,1], \\ \qquad \mathbf{[3,1,1,1]}
	&\qquad \mathbf{[4,2]} \qquad [4,1,1] \qquad \mathbf{[5,1]} \qquad [6]. 
\end{align*}
The bolded types are those which correspond to even partitions, and so
correspond to the conjugacy classes of $A_6$. Recall that these are the types
whose number of entries have the same parity as $6$ (i.e. these are
the types with an even number of rows). Suppose $\sigma \in S_n$ has
type $[a_1, \cdots, a_k]$ then the parity of $\sigma$ is 
$(a_1 - 1) + \cdots (a_k -1)$, since each $a_i$ denotes the length of
a cycle which composes $\sigma$. Now notice 
$(a_1 -1) + \cdots (a_k -1) = \sum_i a_i - \sum_{i=1}^{k} (-1) = 6 -
2\ell = 6 - 2\ell$ is even. And so indeed the chosen permutations give
the conjugacy classes of $A_6$. 

However, we have a bit more counting to do. Recall that a conjugacy
$[\sigma] \subseteq S_n$ splits into two conjugacy classes in $A_n$
exactly when the type of $\sigma$ consists of distinct odd numbers,
and otherwise it splits into a single class in $A_n$. 
In our case we have $[3,3]$ and $[5,1]$ split into two clases in
$A_n$. Hence, overall we have $1 + 1 + 2 + 1 + 1 + 2 = 8$ conjugacy
classes in $A_6$. 

Next we determine the sizes of each conjugacy class in $A_6$. 
Note that the classes not of type $[3,3]$ and $[5,1]$ have the same
size as the corresponding classes in $S_n$. The classes of type
$[3,3]$ and $[5,1]$ split into two classes of equal sizes in $A_6$. 
Recall that the class type gives the sizes of the cycles in cycle
decomposition of $\sigma \in [\sigma]$. And so, we can determine the
size of each class by counting each distinct way of writing a
permutation with the given types.
For example, $[2,2,1,1]$ corresponds to $\sigma = (a_1,
a_2)(b_1b_2)(c_1)(d_1)$ where $a_i, b_i, c_i, d_i \in [n]$.
There are $6!$ ways to populate these numbers, but then we have
equivalent permutations given by cycling the elements in $(a_1,a_2)$
and $(b_1, b_2)$, another equivalence given by interchanging the
cycles, and a final equivalence given by interchanging the two
trivial cycles. We do not need to consider any equivalence given by
interchanging the positions of the $2$-cycles and the trivial cycles,
since this was included in our enumeration of the partitions of $6$,
by definition.
And so the number of elements in the class of type
$[2,2,1,1]$
is given by $\frac{6! = 720}{2 \cdot 2 \cdot 2 \cdot 2} = \frac{720}{16}$. 

A similar kind of counting gives us the following data. In the
following $\abs{[t_i]}$ means the number of elements in the
conjugacy class whose type is given by $[t_i]$.
\begin{align*}
	\abs{[1,1,1,1,1,1,1]} = 1 \qquad
	\abs{[2,2,1,1]} &= \frac{720}{16} \qquad
	\abs{[3,3]} = \frac{720}{3 \cdot 3 \cdot 2} \cdot \frac{1}{2} = \frac{720}{36} \qquad \\
	\abs{[3,1,1,1]} = \frac{720}{3 \cdot 3!} = \frac{720}{18} \qquad
	\abs{[4,2]} &= \frac{720}{4 \cdot 2} = \frac{720}{8} \qquad
	\abs{[5,1]} = \frac{720}{5} \cdot \frac{1}{2} = \frac{720}{10}
\end{align*}
Here the classes with type $[3,3]$ and $[5,1]$ in $S_n$ split into two
distinct equal sized classes in $A_6$ and so we have denoted the size
of each split class in the data above. Then we can write the class
formuala
\[
	1 + 45 + 2(20) + 40 + 90 + 2(72) = 360 = \abs{A_6}
\]
Showing that we have counted the size of our conjugacy classes
correctly.

Lastly, we write the elements of our classes. First consider the
classes which do not split in $A_6$. These classes have the same
elements in $A_6$ as they do in $S_6$. The type of the class tells us
the cycle decomposition of its elements. For example the class whose
type is $[2,2,1,1]$ contains even permutations whose cycle
decomposition is $\sigma = (a_1, a_2)(b_1, b_2)(c_1)(d_1)$ for $a_i,
b_i, c_i, d_i \in [n]$ and distinct. Since permutation type is
preserved by conjugation, this argument is well defined for a given
conjugacy class.
The same reasoning applies to the classes whose type is
$[1,1,1,1,1,1], [2,2,1,1], [3,1,1,1],$ or $[4,2]$.

The classes in $S_6$ whose type is $[3,3]$ or $[5,1]$ split into two distinct
equal size classes in $A_6$. 
\wg{The two split classes must end up containing all the permutations 
whose type is $[3,3]$ or $[5,1]$ in $S_6$. Those elements have a
similar form to what's argued above.}

\wg{
However, for these split classes, there must be
representatives in $S_6$ which are acquired under conjugation by an odd cycle,
hence giving us two classes in $A_6$. For $[3,3]$ I could argue one
class in $A_6$ contains $\sigma = (1,2,3)(4,5,6)$, and then the other class has
a representative given by conjugating $\sigma$ with some odd
permutation, and then all other representatives are given by
conjugating those elements with even permutations. But this seems less
concrete than I would like.}

\end{solution}

\newpage

%------------------------- Problem 4 -----------------------

\begin{problem}
	\includegraphics[scale=0.8]{4.png}
	\hfill
\end{problem}

\begin{solution}
Recall that a subgroup $H \leq G$ is normal if the set of
its left cosets are equal to the set of its right cosets, by
definition of normality. That is if
$\{g H : g \in G\} = \{H g : g \in H\}$. 

Recall that $[G: H] = 2$ means that $H$ has exactly two left/right
cosets\footnote{Recall that there's a bijection between the left
cosets of a subgroup and the right cosets of a subgroup, and so, this
is a well-defined quantity.}. Since $e \in G$, $e \cdot H = H$, and $H
\cdot e = H$, it
must be that $H$ is
one of the left cosets of $H$ and also one of the right cosets of $H$.
Also recall that the left/right cosets of a subgroup partition $G$ as
a set. It follows then that the non-$H$
left coset is $G \setminus H$, and the non-$H$ right coset is also $G
\setminus H$.

That is, we have shown that the left cosets of $H$ are equal to the right cosets of $H$
and so $H$ must be normal. 

\wg{Lang proves this using a lot of machinary of the orbits of group
actions and the kernel of group actions. That all seems more technical
than what I've done here. So I'm a bit worried that I've missed
something.}
\end{solution}

\newpage


%------------------------- Problem 5 -----------------------

\begin{problem}
	\includegraphics[scale=0.8]{5.png}
	\hfill
\end{problem}

\begin{solution}
Note that the same exercise is given in Aluffi $IV.1.18$, I will be
using the hint given in that version of this problem.
The hint gives the following information: $(1)$ The action of $G$ on $X$ is
isomorphic to $G$ acting on $G/H$ where $H = \text{Stab}(x)$ for some
$x \in X$. Here, $G$ acts on $G/H$ by left multiplication. $(2)$ If $H
\lneq G$ for finite $G$ then $G$ is not the union of conjugates of
$H$. \wg{If time, come back
and prove at least the second one.}

First, we show that, since $G$ acts transitively on $X$,
if $x \in X$ then $\text{Stab}(x)$ is conjugate to
$\text{Stab}(y)$ for all $y \in X$. 
Suppose $g$ stabilizes $x$, that is $gx = x$. Since $G$ acts
transitively on $X$ we have $y = \overline g x$ for some $\overline g
\in G$. Consider the following
\begin{align*}
	g x &= x \\
	g(\overline g y) &= \overline g y \\
	(\overline g\inv g \overline g) &= y.
\end{align*}
That is, a conjugate of $g$ stabilizes $y$ for each $g \in
\text{Stab}(x)$. In other words, $g \cdot \text{Stab}(x) \cdot g\inv \subseteq
\text{Stab}(y)$ for some $g \in G$. A similar calculation gives the
reverse inclusion.

Now suppose that, for contradiction, every $g \in G$ fixes some $x \in
X$. Then $G \subseteq \bigcup_{x \in X} \text{Stab}(x)$, and so, 
$G = \bigcup_{x \in X} \text{Stab}(x)$. Let $H = \text{Stab}(x')$ for
some $x' \in X$. Notice that $H \lneq G$ because $\abs{X} \geq 2$ and
because $G$ acts transitively on $X$.
If $H$ were not proper then we would have $g x' = x'$ for all $g \in G$, however, we
know there exists some $x'' \neq x' \in X$ and then no $g \in G$ 
satisfies $g x' = x''$, a contradiction. 
From the preceding paragraph, we can rewrite $\text{Stab}(x) = g_x H g_x\inv$ for some $g_x
\in G$. Then we have \[
G = \bigcup_{x \in X} \text{Stab}(x) = \bigcup_{x \in X} g_x H
g_x\inv, 
\]
a contradiction of fact $(2)$ above. 




\end{solution}

\newpage

%------------------------- Problem 6 -----------------------

\begin{problem}
	\includegraphics[scale=0.8]{6.png}
	\hfill
\end{problem}

\begin{solution}
\wg{Big apologies}
\end{solution}

\newpage

%------------------------- Problem 7 -----------------------

\begin{problem}
	\includegraphics[scale=0.8]{7.png}
	\hfill
\end{problem}

\begin{solution}
\begin{itemize}
\item First, we can view $N_i \leq G_i$ as say $p_1(N_1 = \ker p_2 =
\{(x,e_{G_2}) \in H\}) \leq
G_1$. This is indeed a subgroup of $G_1$ becuase if $x, y \in
p_1(N_1)$ then this means $(x, e), (y, e) \in N_2$ and so, since $H$
is a subgroup, $(x +_{G_1} y, e)
\in N_1$ hence $x +_{G_1} y \in p(N_1)$. And a similar argument holds for inverses and
the identity. 
\wg{If we get time, write out the arg to show that it contains identity and inverses.}
Put another way, $p_1(N_1) \leq G_1$ since $N_1 \leq H$. 
The same reasoning shows that we can view $N_2 \leq G_2$. 

Now we show that $N_1 \unlhd G_1$.
Let $x \in N_1 \leq G_1$ and let $g \in G_1$. Since $p_1: H \to G_1$
is surjective there exists $(g, y) \in H$ and $(g, y)\inv = (g\inv, y\inv) \in H$
since $H$ is a subgroup. Lastly $(x,e) \in H$ by definition of $\ker
p_2$. 
Now consider \[
	(g, y) \cdot_{(G_1 \times G_2)} (x, e) \cdot_{(G_1 \times G_2)} (g\inv, y\inv) 
	= (g x g\inv, e) \in H
\]
since $H$ is closed under $\cdot_{G_1 \times G_2}$. But then we have
shown $g x g\inv \in N_1 \leq G_1$. Thus, $N_1$ is closed under
conjugation and is normal in $G_1$. A very similar argument holds to
show that $N_2 \normal G_2$. 

\item Let $\overline H$ be the image of $H$ in $G_1/N_1 \times
G_2/N_2$. We want to show that $(\overline x, \overline y) \in
\overline H$ associates elements $\overline x \in G_1/N_1$ to
$\overline y \in G_2 /N_2$, as a function, in a bijective manner, and
as a group homomorphism.

To be clear, $\overline H = \overline H_1 \times \overline H_2$ where
$\overline H_i = \im (H \twoheadrightarrow^{p_i} G_i
\twoheadrightarrow^{\pi_i} G_i / N_i)$. 
First we show that $\overline H$ defines a function. That is, we need
to show that there is exactly one element of the form $(\overline x,
-) \in \overline H$ for each $\overline x \in G_1/N_1$. 
Notice that if $(x, y_1), (x, y_2) \in H$ then we have $y_1 - y_2 \in
N_2$ and so $\overline y_1 = \overline y_2 \in G_2 / N_2$. That is,
any elements of $G_2$ which are associated with $x$ in $H$
end up in the same class in $G_2 / N_2$.
And likewise for any $x' \in G_1$ with $\overline{x'} = \overline x$.
It then follows that there is
at most one element of the form $(\overline x, -) \in \overline H$ for
each $\overline x \in G_1 / N_1$. Moreover, $H
\twoheadrightarrow G_1 \twoheadrightarrow G_1/N_1$ is surjective since
it is the composition of surjective maps. It then follows that for all
$\overline x \in G_1/N_1$ there is some element $(\overline x, -) \in
\overline H$. Thus $\overline H$ defines a function $f : G_1/N_1 \to
G_2/N_2$. 

Next we show that $f$ is bijective. First notice that $H
\twoheadrightarrow G_2 \twoheadrightarrow G_2/N_2$ again is
surjective. Thus for each $\overline y \in G_2/N_2$ we have some $(-,
\overline y) \in \overline H$. That is, $f$ is surjective.
Now notice that if $(x_1, y), (x_2, y) \in H$ then we have $x_1 - x_2
\in N_1$. Hence, again, all elements which are associated to $y$ in
$H$ end up in the same class in $G_1 / N_1$. And likewise for $x' \in
G_1$ which associate to some $y' \in G_2$ such that $\overline{y'} =
\overline y$. 
That is, there is at most
one element of the form $(-, \overline y) \in \overline H$. That is,
$f$ is injective.

Lastly, $f$ is a group homomorphism because $\overline H$ is a
subgroup of $G_1 / N_1 \times G_2 / N_2$; this follows since $H$ is a
subgroup of $G_1 \times G_2$.
\wg{There's some unpacking and deatil checking to do here, but I
currently believe this follows from unpacking all the definitions of
the objects. }

\end{itemize}
\end{solution}

\newpage

\end{document}
